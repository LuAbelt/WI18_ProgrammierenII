\documentclass[a4paper,
			   fontsize=12pt]{article}

\usepackage[utf8]{inputenc}
\usepackage[english, ngerman]{babel}
\usepackage[T1]{fontenc}
%\usepackage{polyglossia}
%\usepackage{fancyhdr}
\usepackage{geometry}
\usepackage{xcolor}
\usepackage{listings}
\usepackage[ngerman]{struktex}
\usepackage{enumitem}

% Custom Colors
\definecolor{pblue}{rgb}{0.13,0.13,1}
\definecolor{pgreen}{rgb}{0,0.5,0}
\definecolor{pred}{rgb}{0.9,0,0}
\definecolor{pgrey}{rgb}{0.46,0.45,0.48}
\definecolor{javared}{rgb}{0.6,0,0} % for strings
\definecolor{javagreen}{rgb}{0.25,0.5,0.35} % comments
\definecolor{javapurple}{rgb}{0.5,0,0.35} % keywords
\definecolor{javadocblue}{rgb}{0.25,0.35,0.75} % javadoc

% Custom LST styles
\lstdefinestyle{java}{
	language=Java,
	basicstyle=\linespread{0.8}\ttfamily,
	breaklines=true,
	keywordstyle=\color{javapurple}\bfseries,
	stringstyle=\color{javared},
	commentstyle=\color{javagreen},
	morecomment=[s][\color{javadocblue}]{/**}{*/},
	numbers=left,
	numberstyle=\tiny\color{black},
	stepnumber=1,
	numbersep=10pt,
	tabsize=4,
	showspaces=false,
	showstringspaces=false,
	morekeywords={enum},
	postbreak=\mbox{\textcolor{red}{$\hookrightarrow$}\space},
}

%\geometry{a4paper,left=15mm,right=15mm,top=20mm,bottom=20mm}
%\pagestyle{fancy}
%\lhead{Lukas Abelt}
%\chead{WWI218 - Programmieren II}
%\rhead{\today}
%\cfoot{\thepage}

%\setlength{\headheight}{23pt}
\setlength{\parindent}{0.0in}
%\setlength{\parskip}{0.0in}

\begin{document}

\section*{Kurztest WWI218 Programmieren II - Algorithmen}
\textbf{Name:} 
\vspace{0,5cm}
\subsection*{Allgemeine Informationen}
Alle Aufgaben ergeben insgesamt \textbf{25 Punkte}. Maximal zu erreichen sind \textbf{20 Punkte}. Somit müssen nicht alle Aufgaben bearbeitet werden um die volle Punktzahl zu erreichen.
Die Wahl der zu bearbeitenden Aufgaben steht Ihnen frei. Jedoch wird empfohlen, die Aufgaben 1-3 sowie eine der Aufgaben 4.1 oder 4.2 zu bearbeiten.

Die Bearbeitungszeit beträgt \textbf{30 Minuten}. Die Bearbeitung der Aufgaben erfolgt \textbf{ausschließlich} auf den ausgehändigten Aufgabenblättern. Bei Benutzung der Rückseite ist die Aufgabennummer mit zu vermerken.

Dieser Kurztest geht zu \textbf{20\%} in die Gesamtbewertung der Vorlesung ein.

\subsection*{Bewertung}
\begin{tabular}{|c|c|c|}
\hline
\textbf{Aufgabe}&\textbf{Erreichte Punktzahl}&\textbf{Maximale Punktzahl}\\
\hline
Aufgabe 1 & & 4
\\\hline
Aufgabe 2 & & 8
\\\hline
Aufgabe 3 & & 4
\\\hline
Aufgabe 4.1 & & 4
\\\hline
Aufgabe 4.2 & & 4
\\\hline
Aufgabe 5 & & 1
\\\hline
\textbf{Gesamt:} & & 20
\\\hline
\end{tabular}

\vspace{2cm}
\hline


\small{Datum, Ort \hspace{8cm} Bewertung, Unterschrift Dozent}

\newpage

\subsection*{Aufgabe 1 (4 Punkte)}
In der Vorlesung haben sie zwei Messgrößen zur Bestimmung der Komplexität kennengelernt. Nennen Sie die beiden Messgrößen und erläutern sie kurz den Unterschied!

\vspace{6cm}

\subsection*{Aufgabe 2 (8 Punkte)}
Folgend ist ein Codeausschnitt gezeigt, der für eine gegebene Zahl $n$ ermittelt, ob es sich um eine Primzahl handelt oder nicht. Zeigen Sie, ob der Algortihmus korrekt umgesetzt ist.
\begin{itemize}
	\item Funktioniert der Algorithmus korrekt?
	\item Falls der Algorithmus nicht korrekt funktioniert, nennen Sie den Grund und geben Sie an, welche Änderung notwendig ist, damit der Algorithmus korrekt funktioniert.
	\item Bestimmen Sie $O$
	\item Stellen Sie den (ggf. korrigierten) Algorithmus als Struktogramm oder Programmablaufplan dar.
\end{itemize}

\lstset{style=java}
\begin{lstlisting}
public boolean isPrime(int n){
	for(int i=1;i<n;i++){
		if(n%i==0){
			return false;
		}
	}
	return true;
}
\end{lstlisting}

\vspace{7,5cm}

\subsection*{Aufgabe 3 (4 Punkte)}
Im folgenden Codeausschnitt wird ein Algorithmus gezeigt der verschiedene Laufzeiten haben kann. Übergeben wird ein Integer Array (\texttt{array}) und die Größe dieses
Arrays(\texttt{n}).
\begin{itemize}
	\item In welchem Fall tritt der Worst-Case und wann der Best-Case ein? (In Hinblick auf die Laufzeit)
	\item Bestimmen Sie $O$
\end{itemize}  

\begin{lstlisting}
public void func(int[] array, int n){
    for(int i=0;i<n;i++){
		if((array[i]%2)==0){	//3 Operationen, n mal
			for(int j=0;j<n;j++){
				System.out.println(array[i]*array[j])
			}
        }else{
			System.out.println(array[i])
        }
    }
}
\end{lstlisting}

\newpage
\subsection*{Aufgabe 4.1 (4 Punkte)}
Analysieren Sie den folgenden Codeausschnitt. Hinweis: $n$ beschreibt hierbei die Länge des übergebenen Arrays. Die \texttt{System.out.println()} Funktion wird als eine Operation gewertet.
\begin{itemize}
	\item Bestimmen Sie $\tau(n)$ und $\tau(4)$
	\item Beschreiben Sie kurz die Aufgabe des Algorithmus
\end{itemize}

\begin{lstlisting}
public void func(int[] array, int n){
	for(int i=0;i<n;i++){
		System.out.println(array[i]*array[i])
	}
}
\end{lstlisting}

\vspace{5cm}
\subsection*{Aufgabe 4.2 (4 Punkte)}
Im folgenden wird ein Struktogramm für einen Algorithmus gezeigt.

\begin{centernss}
	\begin{struktogramm}(100,50)
		\while{Von \( i=n-1 \) bis \( 1 \) , Schrittweite 1}
			\ifthenelse{3}{3}
			{ \( n\%i==0 \) }{True}{False}
			\assign{Gib \( i \) zurück}
			\change
			\endif
		\whileend
		\assign{Gib \( n \) zurück}
	\end{struktogramm}
\end{centernss}

\begin{itemize}
	\item Welche Aufgabe hat der Algorithmus?
	\item Setzen Sie das Struktogramm in Java-Code um. Nutzen Sie dafür die unten gegebene Vorlage.
\end{itemize}

\vspace{2cm}

Hinweis: Die Operation "`Einlesen"' wird durch das übergeben von Funktionsargumenten realisiert und das zurückgeben von Werten über das \texttt{return} statement.

\begin{lstlisting}
public        func(             ){






}
\end{lstlisting}

\subsection*{Aufgabe 5 (1 Punkt)}
Welche ist die beste IDE? (Nicht zutreffendes bitte streichen)
\begin{itemize}
	\item Eclipse im Default Theme
	\item Eclipse im Dark Theme
	\item IntelliJ im Default Theme
	\item IntelliJ im Dark Theme
	\item vim
\end{itemize}
\end{document}