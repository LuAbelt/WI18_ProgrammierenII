\begin{frame}{Eventhandling}{Grundlagen für Java}
    \begin{itemize} 
        \item Jede Komponente kann Ereignisse auslösen
        \item Auf diese kann reagiert werden
        \item Dafür müssen sogenannte \textit{Event Listener} implementiert werden
        \item Diese Registrieren sich dann auf die Event Auslöser ("`event sources"' - Also die einzelnen Komponenten)
        \item Ein Event Listener kann auf beliebig viele Auslöser registriert werden
        \item Für einen Auslöser können beliebig viele Listener registriert werden
    \end{itemize}
\end{frame}

\begin{frame}{Ablauf}{Beim Auslösen von Events (Vgl. \cite{ullenboom2018java} S. 806f)}
    \begin{itemize}
        \item Bei Auslösen des Events werden von der Komponente die registrierten Listener "`benachrichtigt"'
        \item Diese führen dann ihre implementierten Aktionen aus
        \item Der Mechanismus nutzt also das \textit{Delegate Entwurfsmuster}
        \item Swing stellt einige \texttt{EventListener} Interfaces bereit, unter anderem:
        \begin{itemize}
            \item \texttt{ActionListener} -- Aktivieren einer Schaltfläche bzw. Menüs (z.B. Klicken eines Buttons)
            \item \texttt{WindowListener} -- Reagiert auf Änderungen des Fensters
            \item \texttt{MouseListener} -- Reagiert auf Mausklicks
            \item \texttt{MouseMotionListener} -- Reagiert auf Bewegungen der Maus
        \end{itemize}
    \end{itemize}
\end{frame}

\begin{frame}{Registrieren}{Von Event Listenern}
    \begin{itemize}
        \item Je nach Komponente werden ggf. nicht alle Events unterstützt
        \item Einige werden jedoch von allen unterstützt
        \item Um einen neuen Listener wird auf der Komponente eine Methode aufgerufen:
        \begin{itemize}
            \item \texttt{addXXXListener(XXXListener l)} -- "`XXX"' zu ersetzen durch den Eventtyp (Bspw. "`Action"', "`Mouse"' etc.)
        \end{itemize}
        \item Zum Entfernen existiert analog eine Methode zum Entfernen:
        \begin{itemize}
            \item \texttt{removeXXXListener(XXXListener l)}
        \end{itemize}
    \end{itemize}
\end{frame}

\begin{frame}{Implementieren}{Von Event Listener(Vgl. \cite{ullenboom2014java} S. 1086f)}
    \begin{itemize}
        \item Für die Implementierung der Interfaces gibt es verschiedene Möglichkeiten:
        \begin{itemize}
            \item Die Komponente implementiert selbst das Interface und ist sein eigener Listener
            \item Listener Interface wird durch eine externe Klasse realisiert
            \item Listener wird über eine interne Anonyme Klasse implementiert
            \item Listener wird als \textit{Lambda Ausdruck} definiert
        \end{itemize}
        \item Alle Methoden haben eigene Vor- und Nachteile
    \end{itemize}
\end{frame}

\begin{frame}{Implementierung}{Direkt in Komponente}
    \begin{itemize}
        \item Die erweiterte Klasse implementiert die benötigten Interfaces selbst
        \item Beispielsweise könnte eine eigene Unterklasse von \texttt{JButton} auch direkt das \texttt{ActionListener}
        \item Komponente setzt sich dann selbst als Listener
        \begin{itemize}
            \item Durch Übergeben von \texttt{this} in der \texttt{addXXXListener} Methode
            \item Kann bspw. automatisch in definiertem Konstruktor passieren
        \end{itemize}
        \item Nützlich, wenn zum Beispiel ein Default-Verhalten für eine Komponente definiert werden soll
    \end{itemize}
\end{frame}

\begin{frame}[fragile]{Codebeispiel}{Von Button mit eigenem Listener}
\lstset{style=java}
\begin{lstlisting}
public class MyButton extends JButton implements ActionListener {
  public MyButton(){
    super();
    addActionListener(this);
  }
  
  @Override
  public void actionPerformed(ActionEvent e) {
    setText("Handled by myself!");
  }
}
\end{lstlisting}
\end{frame}

\begin{frame}{Anonyme Klassen}{Begriffsklärung}
    \begin{itemize}
        \item Werkzeug um eine Klasse gleichzeitig zu deklarieren und zu definieren
        \item Klasse wird innerhalb von Methoden quasi "`on-the-fly"' definiert
        \item Realisieren meist die Implementierung eines bestimmten Interfaces bzw. einer speziellen Unterklasse
        \item Vergleichbar mit lokalen Klassen $\rightarrow$ Jedoch ohne Namen
        \item Verwendung: Wenn die spezifische Klasse nur einmal benötigt wird
        \item Syntax: Ähnlich wie ein normaler Konstruktor:
        \begin{itemize}
            \item \texttt{new SomeClass()\{ /* Klassendefinition */\};}
        \end{itemize}
    \end{itemize}
\end{frame}

\begin{frame}{Anonyme Klassen}{Als Listener}
    \begin{itemize}
        \item Anonyme Klasse wird bei Aufruf der \texttt{addXXXListener()} Methode deklariert
        \item Und zwar als Übergabeparameter
        \item Benötigte Methoden des entsprechenden Listeners werden dann überschrieben
        \item Vorteil gegenüber der direkten Definition in der Komponenten:
        \begin{itemize}
            \item Es muss keine eigene Komponenten-Klasse geschrieben werden
            \item Anonyme Klassen können für Standard-Komponenten erstellt werden
        \end{itemize}
        \item Nachteile von anonymen Listenern:
        \begin{itemize}
            \item Nicht direkt wiederverwendbar für andere Komponenten
            \item Entfernen des Listeners ist nicht trivial
        \end{itemize}
    \end{itemize}
\end{frame}

\begin{frame}[fragile]{Anonyme Listener}{Codebeispiel}
\lstset{style=java}
\begin{lstlisting}
JButton btn = new JButton();
btn.addActionListener(new ActionListener() {
    @Override
    public void actionPerformed(ActionEvent e) {
      btn.setText("Handled by anonymous class");
    }
  });
\end{lstlisting}
\end{frame}

\begin{frame}{Lambda Expressions}{Begriffsklärung}
    \begin{itemize}
        \item Bildet zum Teil funktionale Programmierung in Java ab
        \item Lambda Ausdrücke sind im Grunde Funktionen
        \item ...die jedoch zu keiner Klasse gehören
        \item Lambda Ausdrücke können wie Objekte zwischen Klassen und Methoden ausgetauscht werden
        \item Syntax:
        \begin{itemize}
            \item \texttt{ (Argumente) -> \{ /* Operationen */ \}}
            \item \texttt{ (int x, int y) -> x+y}
            \item \texttt{ () -> 42 }
            \item \texttt{ (String s) -> \{ System.out.println(s); \}}
        \end{itemize}
        \item Häufig verwendet in Verbindung mit \textit{Collections}
    \end{itemize}
\end{frame}

\begin{frame}{Eventhandling}{Mit Lambda Expressions}
    \begin{itemize}
        \item Lambda Ausdrücke können \textit{funktionale Interfaces} ersetzen
        \item In diesem Fall wird statt einer Klasse ein Lambda Ausdruck übergeben
        \item Dadurch teilweise auch für Listener nutzbar
        \item Beispielsweise sind funktionale Interfaces:
        \begin{itemize}
            \item \texttt{ActionListener}
            \item \texttt{ChangeListener}
        \end{itemize}
        \item Vor- und Nachteile im Grunde wie bei anonymen Klassen
        \item Nur weniger Code als bei diesen
    \end{itemize}
\end{frame}

\begin{frame}[fragile]{Codebeispiel}{Für ein Lambda Eventhandler}
\lstset{style=java}
\begin{lstlisting}
JButton btn = new JButton();
btn.addActionListener( e -> btn.setText("Handled by Lambda Expression"));
\end{lstlisting}
\end{frame}

\begin{frame}{Listenerklassen}{Getrennt von den Komponenten}
    \begin{itemize}
        \item Häufig werden Listener als eigene Klassen implementiert
        \item Neue Klasse implementiert dann das gewünschte Eventhandling-Interface
        \item Vorteile:
        \begin{itemize}
            \item Listener sind wiederverwendbar
            \item Logische Trennung von Model und View
        \end{itemize}
        \item Nachteile:
        \begin{itemize}
            \item Mehr Klassen im Projekt $\rightarrow$ Übersichtlichkeit
            \item Wenn die Komponente manipuliert werden soll (z.B. \texttt{setText}) so muss eine Referenz auf diese gespeichert werden
        \end{itemize}
    \end{itemize}
\end{frame}

\begin{frame}[fragile]{Implementierung}{Der Listener Klasse}
\lstset{style=java}
\begin{lstlisting}
public class MyListener implements ActionListener{
  private JButton buttonRef;
  
  public MyListener(JButton btn){
    buttonRef = btn;
  }
  
  @Override
  public void actionPerformed(ActionEvent e) {
    buttonRef.setText("Handled by external class!");
  }
}
\end{lstlisting}
\end{frame}

\begin{frame}[fragile]{Implementierung}{Verwenden des Listeners}
\lstset{style=java}
\begin{lstlisting}
/* */
JButton btn = new JButton();
MyListener listener = new MyListener(btn);
btn.addActionListener(listener);
/* */
\end{lstlisting}
\end{frame}