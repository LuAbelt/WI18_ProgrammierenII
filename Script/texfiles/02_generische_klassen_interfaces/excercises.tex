\begin{frame}{Aufgaben 1}{KeyValue-Pair}
    Entwickelt eine \texttt{Pair}-Klasse, die zwei Instanzvariablen (\texttt{Key} und \texttt{Value}) speichert. Definiert die Instanzvariablen über die Nutzung von Generics.
    Implementiert einen Konstruktor, mit dem ein \texttt{KeyValue}-Objekt erzeugt wird und die beiden Instanzvariablen initialisiert. Baut Methoden ein, um die Werte unabhängig voneinander abzurufen und zu setzen.
\end{frame}

\begin{frame}[allowframebreaks]{Aufgabe 2}{Array-Wrapper Klasse}
    Entwickelt eine Wrapper Klasse für Arrays. Die Klasse soll ein Array eines generischen Typs speichern.

    Außerdem sollte die Klasse folgende Funktionen implementieren:
    \begin{enumerate}
        \item Konstruktor dem die initialen Daten für das Array übergeben werden
        \item \texttt{printData} - Gibt den Inhalt des Arrays auf der Kommandozeile aus
        \item \texttt{getData}/\texttt{setData} - Setzen das gespeicherte Array bzw. geben dieses zurück
        \item \texttt{getElement}/\texttt{setElement} - Setzen ein bestimmtes Element (über Angabe des Indizes) bzw. geben dieses zurück
        \item \texttt{contains} - Überprüft ob ein bestimmtes Element im Array vorhanden ist
        \item \texttt{countOccurences} - Zählt die Vorkommen eines bestimmten Elements im Array
        \item \texttt{invertArray} - Kehrt die Reihenfolge der Daten im gespeicherten Array um
    \end{enumerate}
\end{frame}

\begin{frame}{Aufgabe 3}{Aufgabe 3: Erweiterung der Wrapper-Klasse für \texttt{Number}}
    Erweitert die in 2. entwickelte Klasse so, dass das gespeicherte Array vom Typ \texttt{Number} (oder einer der Subtypen) sein muss.

    Implementiert zusätzlich folgende Methoden:
    \begin{enumerate}
        \item \texttt{getSum} - Ermittelt die Summe aller Elemente im Array
        \item \texttt{getProduct} - Ermittelt das Produkt aller Elemente im Array
        \item \texttt{getDifference} - Ermittelt die Differenz aller Elemente im Array
        \item \texttt{getMin}/\texttt{getMax} - Ermittelt Mini- bzw. Maximum aller Elemente
        \item \texttt{getAverage} - Bestimmt das arithmetische Mittel
    \end{enumerate}

    Hinweis: Nutzt hierfür die \texttt{doubleValue()} Methode der \texttt{Number} Klasse
\end{frame}

\begin{frame}[allowframebreaks]{Aufgabe 4}{Erweiterung der Wrapper Klasse für \texttt{Comparable}}

    Erweitert die in 2. entwickelte Klasse so, dass der Typ des gespeicherten Arrays das \texttt{Comparable} Interface implementieren muss.

    Implementiert zusätzlich folgende Methoden:
    \begin{enumerate}
        \item \texttt{getMin}/\texttt{getMax} - Ermittelt Mini- bzw. Maximum aller Elemente
        \item \texttt{allSmaller} - Überprüft, ob alle Werte im Array kleiner als ein gegebener Wert sind
        \item \texttt{allGreater} - Überprüft, ob alle Werte im Array größer als ein gegebener Wert sind
        \item \texttt{allInRange} - Überprüft, ob alle Werte im Array zwischen zwei gegebenen Werten liegen
        \item \texttt{allOutOfRange} - Überprüft, ob alle Werte im Array außerhalb von zwei gegebenen Werten liegen
        \item \texttt{clamp} - Es wird eine untere und eine obere Grenze übergeben. Alle Werte im Array die kleiner als die untere Grenze sind, werden auf diesen Wert gesetzt. Analog geschieht das mit der oberen Grenze
    \end{enumerate}

    Hinweis: Nutzt hierfür die \texttt{compareTo(T)} Methode des \texttt{Comparable<T>} Interfaces
\end{frame}

\begin{frame}[allowframebreaks]{Aufgabe 5}{Wildcards}
Entwickelt eine Klasse, die verschiedene statische Methoden für Listenoperationen zur Verfügung stellt. 

Nutzt hierbei zur Umsetzung Wildcards.
Implementiert mindestens die folgenden Funktionen:

\begin{enumerate}
    \item Eine Methode, die alle Elemente einer übergebenen Liste ausgibt (Ähnlich wie \texttt{printData()} aus Aufgabe 2)
    \item Eine Methode, die für eine beliebige Liste vom Typ \texttt{Number} die Summe berechnet
    \item Eine Methode, die zwei Listen vom Typ \texttt{Number}(Oder der Unterklassen) zu einer einzigen kombiniert. (Hinweis: sowohl die beiden Eingabelisten, 
    als auch die Ausgabeliste sollen als Parameter übergeben werden)
    \item Eine Methode, die für zwei Listen eines beliebigen \texttt{Number} Typs die Elementweise Summe berechnet und diese in einer neuen Liste speichert.
\end{enumerate}
\end{frame}