\begin{frame}{Java-Klassenbibliothek}{Kurzüberblick}
	\begin{itemize}
		\item Java bietet Vielzahl an "`fertigen"' Klassen
		\item Zusammengefasst in Packages
		\item Diese implementieren Standardfunktionalitäten wie z.B.
		\begin{itemize}
			\item Ein- und Ausgabefunktionalitäten
			\item Grafische Oberflächen
			\item Netzwerkkommunikation
			\item Datum- und Zeit, Internationalisierung
			\item Und viele mehr...
		\end{itemize}
	\end{itemize}
\end{frame}

\begin{frame}[allowframebreaks]{Wichtige Packages}{... der Klassenbibliothek}
Die wichtigsten Standardpackages im schnellen Überblick:
	\begin{itemize}
		\item \textit{java.lang}: Integriert die fundamentalen Klassen, die in der Regel immer zur Java Entwicklung benötigt werden wie zum Beispiel \texttt{String} \texttt{Object} oder auch die Wrapper Klassen der primitiven
		Datentypen (\texttt{Integer}, \texttt{Boolean}, \texttt{Double} usw.). \textbf{Muss nicht explizit importiert werden!}
		\item \textit{java.util}: Häufig benötigte Klassen, wie Listenstrukturen (\texttt{List}, \texttt{Stack}), Klassen zur Verarbeitung von Datum und Uhrzeit (\texttt{Calendar}) oder Zufallszahlengeneratoren (\texttt{Random})
		\item \textit{java.io}: Klassen zur Ein- und Ausgabe über Streams
		\item \textit{java.net}: Klasse zur Implementierung von Netzwerkkommunikation
		\item \textit{java.rmi}: Klassen zur Entwicklung verteilter Programme unter Nutzung von Remote Method Invocation
		\item \textit{java.awt}: Grundlegendes Package für die Entwicklung grafischer Oberflächen
		\item \textit{java.swing}: Erweiterte Komponente zur Entwicklung von grafischen Oberflächen. Baut auf \textit{java.awt} auf, bietet jedoch mehr Funktionalität
		\item \textit{javax.crypto} und \textit{java.security}: Klassen zur Umsetzung von sicherheitsrelevanten Aspekte (Zugriffsschutz, Rechteverwaltung etc.)
		\item \textit{java.sql}: Package zur Interaktion mit SQL Datenbanken
	\end{itemize}
    
    Vgl. \cite{ullenboom2007java}
\end{frame}

\begin{frame}{Arbeiten mit der Klassenbibliothek}
	\begin{itemize}
		\item Packages über \texttt{import} in Klasse einbinden
		\item Oracle bietet umfangreiche Dokumentation zu allen Klassen der Standard-API
		\item Schnellster Weg zur Doku:
		\begin{itemize}
			\item In den meisten IDEs sowieso integriert
			\item Sonst Google: "`Java 10/11/12 API \textit{PackageName}"'
		\end{itemize}
	\end{itemize}
\end{frame}

\begin{frame}{API-Dokumentation}{Links}
	\vfill
    \begin{block}{Java 10 API}
		\url{https://docs.oracle.com/javase/10/docs/api/index.html}
	\end{block}
	
    \begin{block}{Java 11 API}
		\url{https://docs.oracle.com/en/java/javase/11/docs/api/index.html}
	\end{block}
	
	\begin{block}{Java 12 API (Aktuell)}
		\url{https://docs.oracle.com/en/java/javase/12/docs/api/index.html}
	\end{block}
	
	
	\vfill
\end{frame}