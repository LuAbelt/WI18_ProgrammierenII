\begin{frame}{Wildcards}{Was ist das jetzt schon wieder?}
    \begin{itemize}
        \item Weiteres Werkzeug für generischen Code, wenn der Typ nicht explizit bekannt sein muss
        \item Dargestellt über \texttt{<?>} im Code
        \item Wildcards können nach "`oben"' oder "`unten"' (Im Sinne der Vererbungshierarchie) eingeschränkt werden
        \item Vereinfacht die Implementierung von Methoden, die Generische Klassen verwenden
        \item Wird \textbf{nicht} verwendet für:
        \begin{itemize}
            \item Erzeugung von Objekten einer generischen Klasse
            \item Als Typargument bei Aufruf generischer Methoden
            \item Zur spezifizierung eines Supertyps
        \end{itemize}
    \end{itemize}
\end{frame}

\begin{frame}{Unbounded Wildcards}
    \begin{itemize}
        \item Schränken den Typ nicht ein
        \item Nützlich, wenn:
        \begin{itemize}
            \item Die Methode sich allein über die Funktionalitäten der \texttt{Object} Klasse realisieren lässt
            \item Die verwendeten Methode der generischen Klasse nicht vom Typ abhängig sind
            \begin{itemize}
                \item Zum Beispiel: \texttt{List.size()} oder \texttt{List.clear()}
            \end{itemize}
        \end{itemize}
    \end{itemize}
\end{frame}

\begin{frame}[fragile]{Unbounded Wildcards}{Codebeispiel}
\lstset{style=java}
\begin{lstlisting}
public static void printList(List<?> list){
    for (Object elem : List){
        System.out.println(elem.toString);
    }
}
\end{lstlisting}
\end{frame}

\begin{frame}{Upper-Bounded Wildcards}{Du und alle unter dir}
    \begin{itemize}
        \item Schränkt die Wildcard auf eine Klasse sowie alle ihre Subklassen ein
        \item Definition erfolgt über die Nutzung des \texttt{extends} Keywords
        \begin{itemize}
            \item Beispiel: \texttt{List<? extends Number>}
        \end{itemize}
        \item Upper-Bounded Wildcards stellen sicher, dass Methoden der Superklasse verwendet werden können
        \item Vereinfachen somit Implementierung von Methoden, die für eine gesamte "`Klassenfamilie"' funktionieren sollen (Beispielsweise alle \texttt{Number} Klassen)
    \end{itemize}
\end{frame}

\begin{frame}[fragile]{Upper-Bounded Wildcards}{Codebeispiel: Summe einer Liste}
\lstset{style=java}
\begin{lstlisting}
public static double sumOfList(List<? extends Number> list){
    double s = 0.;
    for (Number n : list){
        s += n.doubleValue();
    }
    return s;
}
\end{lstlisting}
\end{frame}

\begin{frame}{Lower-Bounded Wildcards}{Du und alle über dir!}
    \begin{itemize}
        \item Schränkt die Wildcard auf eine Klasse und all ihre Superklassen ein
        \item Definition erfolgt über das Keyword \texttt{super}
        \begin{itemize}
            \item Beispiel \texttt{List<? super Double>}
        \end{itemize}
    \end{itemize}
\end{frame}

\begin{frame}[fragile]{Upper-Bounded Wildcards}{Codebeispiel: Summe einer Liste}
\lstset{style=java}
\begin{lstlisting}
public static void addNumbers(List<? super Integer> list) {
    for (int i = 1; i <= 10; i++) {
        list.add(i);
    }
}
\end{lstlisting}
\end{frame}

\begin{frame}{Wildcards}{Guidelines zur Benutzung}
\begin{itemize}
    \item Wildcards sollten nicht in Rückgabewerten verwendet werden
    \item Generell lassen sich die zu verwendenen Wildcards nach Art der Variable unterscheiden:
    \begin{itemize}
        \item \texttt{in} Variable: Upper-Bounded
        \begin{itemize}
            \item Falls ausschließlich Methoden aus \texttt{Object} benötigt werden: Unbounded
        \end{itemize}
        \item \texttt{out} Variable: Lower-Bounded
        \item Wenn die Variable sowohl als \texttt{in} und \texttt{out} verwendet wird: Keine Wildcard nutzen
    \end{itemize}
\end{itemize}
\end{frame}