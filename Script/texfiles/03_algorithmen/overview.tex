\outlineSubframe{Begriffsklärung}

\begin{frame}{Begriffklärung}{Etymologie}
    \begin{itemize}[<+->]
        \item Leitet sich ursprünglich vom persischen Astronomen "`Muhammad Ibn-Musa al-Hwarizmi"' ab
        \begin{itemize}
            \item Schrieb Bücher über das indische Zahlensystem (um 800 n. Chr.)
            \item Im 12. Jh übersetzt ins lateinische
            \item Dabei wurde der Namensbestandteil "`al-Hwarizmi"' in "`Algorismi"' lateinisiert
        \end{itemize}
        \item Durch spätere Überlieferungen wurde der Begriff später als Zusammensetzung betrachtet aus...
        \begin{itemize}
            \item Dem Namen "`Algus-"'...
            \item und dem aus dem griechisch entlehnten "`-rismus"' (Zahl)
        \end{itemize}
        \end{itemize}
\end{frame}

\begin{frame}{Begriffsklärung}{Was bedeutet das jetzt}
    \begin{block}{Formale Definition}
    \textit{Eine Berechnungsvorschrift zur Lösung eines Problems heißt genau dann Algorithmus, wenn eine zu dieser Berechnungsvorschrift äquivalente Turingmaschine existiert, die für jede Eingabe, die eine Lösung besitzt, stoppt.}
    \end{block}
    \pause
    \begin{alertblock}{Oder auch}
    \textit{Ein Algorithmus ist eine domänenunabhängige Beschreibung einer Handlungsvorschrift zur Lösung eines Problems. Eine bestimmte Eingabe wird in eine bestimmte Ausgabe überführt.}
    \end{alertblock}
\end{frame}

\begin{frame}{Begriffsklärung}{Also}
    \begin{itemize}[<+->]
        \item Ist also die Beschreibung eines Programmes oder einer Funktion
        \begin{itemize}
            \item Unabhängig von der verwendeten Programmiersprache!
            \item Source Code direkt ist also kein Algorithmus...
            \item ...aber aus diesem lässt sich der verwendete Algorithmus ableiten und beschreiben
        \end{itemize}
        \item Algorithmen können in verschiedenen Formen dargestellt werden (Mehr dazu im nächsten Kapitel)
    \end{itemize}
\end{frame}

\outlineSubframe{Ziele des Moduls}

\begin{frame}{Ziele}{}
    \begin{itemize}[<+->]
        \item Am Ende des Moduls könnt ihr...
        \begin{itemize}
            \item Einen Algorithmus in eine Implementierung umsetzen
            \item Aus einer Implementierung den Algorithmus ableiten
            \item Die formalen Eigenschaften von Algorithmen kennen
            \item Algorithmen anhand der kennengelernten Methoden zu analysieren
        \end{itemize}
    \end{itemize}
\end{frame}