\begin{frame}{Bubble Sort}{Grundprinzip (Siehe \cite{ottmann2017} S. 89ff)}
    \begin{itemize}
        \item Eine Liste wird Elementweise betrachtet
        \item Jedes Element wird mit seinem Nachfolger verglichen
        \item Ist der Nachfolger kleiner, so werden die Elemente getauscht
        \item Dies wird solange wiederholt, bis die komplette Liste durchlaufen wurde ohne, dass eine Vertauschung durchgeführt wurde
        \item Der Name "`Bubble"' leitet sich davon ab, dass die größten Element sich am oberen Ende der Liste wie eine "`Blase"' sammeln
    \end{itemize}
\end{frame}

\begin{frame}{Insertion Sort}{Praktisches Beispiel}
\end{frame}

\begin{frame}{Vor- und Nachteile}{Siehe \cite{ottmann2017} S. 89ff}
    \begin{itemize}
        \item Wohl mit der simpelste Algorithmus
        \item Jedoch ineffizient $\rightarrow$ Komplexität $O(N^2)$
        \item Auch wenn z.B. Insertion Sort die gleiche Komplexität hat ist dieser in der Regel deutlich schneller
        \item Daher nur wenige sinnvolle praktische Anwendungen:
        \begin{itemize}
            \item Beispielsweise erkennen (und korrigieren) von sehr kleinen Fehlern in "`beinahe sortierten"' Arrays (Anwendung in der Computergrafik)
        \end{itemize}
    \end{itemize}
\end{frame}