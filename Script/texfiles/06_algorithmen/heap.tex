\begin{frame}{Heapsort}{Grundprinzip (Vgl. \cite{fahr:algo}, S. 12ff)}
    \begin{itemize}
        \item Sortiert nicht direkt Listen sondern nur die spezielle Struktur "`Heap"'
        \item Das heißt die Daten müssen entweder in dieser Form vorliegen
        \item ...Oder erst in diese Struktur umgewandelt werden
        \item Heap kann als Binärbaum interpretiert werden
        \item Heapsort besteht aus dem wiederholten Entfernen der Wurzel...
        \item ...Und dem nachfolgenden "`versickern"' der restlichen Element
    \end{itemize}
\end{frame}

\begin{frame}{Heap}{Definition}
    \begin{alertblock}{Heap (Definition als Liste, Vgl. \cite{fahr:algo}, S. 12ff)}
    Eine Folge $F = k_1, k_2, \ldots,k_n$ von $n$ Schlüsseln nennen wir dann Heap, wenn
    $$k_i\le k_{\frac{i}{2}}$$
    \end{alertblock}
\end{frame}

\begin{frame}{Heap}{Definition (Vgl. \cite{fahr:algo}, S. 12ff)}
    \begin{alertblock}{Heap (Binärbaum)}
    Ein Heap ist ein vollständiger Binärbaum, in dem der Schlüssel jedes Knotens mindestens so groß ist wie der Schlüssel seiner Söhne
    \end{alertblock}
\end{frame}

\begin{frame}{Heapsort}{Sortieren (Vgl. \cite{fahr:algo}, S. 12ff)}
\begin{alertblock}{Vorgehen Sortieren}
\begin{itemize}
    \item Gebe den Wurzelknoten des Baumes aus und entferne diesen
    \item Setze das letzte Element im Baum an die Wurzel
    \item Versickere die neue Wurzel im Baum
    \item Wiederhole den Prozess bis der Baum leer ist
\end{itemize}
\end{alertblock}
\end{frame}

\begin{frame}{Heapsort}{Versickern (Vgl. \cite{fahr:algo}, S. 12ff)}
\begin{alertblock}{Vorgehen Versickern}
\begin{itemize}
    \item Vergleiche den Wurzelknoten mit dem größten Kindknoten
    \item Ist der Wurzelknoten kleiner als der größte Kindknoten:
    \begin{itemize}
        \item Vertausche den Kindknoten mit dem Wurzelknoten
        \item Wiederhole dies bis beide Kindknoten kleiner als der Wurzelknoten sind (Bzw. keine Kindknoten mehr vorhanden sind)
    \end{itemize}
\end{itemize}
\end{alertblock}
\end{frame}