\outlineSubframe{Divide and Conquer}
\begin{frame}{Motivation}
    \begin{itemize}
        \item Das Lösen großer Probleme fällt oft schwer
        \begin{itemize}
            \item Sowohl für Menschen
            \item Als auch Computer
        \end{itemize}
        \item Meist ist es einfacher, das Problem in Teilprobleme zu unterteilen
        \item ...und diese separat voneinander zu lösen
        \item Wir machen dies oft ganz automatisch
        \item Bei Algorithmen spricht man hierbei vom \textbf{Divide and Conquer} Verfahren
    \end{itemize}
\end{frame}

\begin{frame}{Simples Beispiel}{Multiplikation}
    $$a\cdot b = \overbrace{b+b+\cdots+b}^{a\text{ mal}}$$
    
    $$a\cdot b = \overbrace{(b+b+\cdots+b)}^{\frac{a}{2}\text{mal}} + \overbrace{(b+b+\cdots+b)}^{\frac{a}{2}\text{mal}} $$
    
    $$a\cdot b = 2\cdot \overbrace{(b+b+\cdots+b)}^{\frac{a}{2}\text{mal}}$$
\end{frame}

\begin{frame}{Grundsätze}{Des Divide and Conquer Verfahrens}
    \begin{itemize}
        \item Ein gegebenes Problem wird aufgeteilt...
        \item ...in zwei (oder mehr) kleinere Teilprobleme
        \item Dies geschieht rekursiv solange...
        \item ...bis sich die Teilprobleme trivial direkt lösen lassen
        \item Am Ende werden die Teilergebnisse zur Gesamtlösung zusammengefügt
        \item Es gibt ein ähnliches Vorgehen, bei dem man das Problem lediglich auf \textit{ein} kleineres Teilproblem reduziert
        \begin{itemize}
            \item Dies nennt man auch \textbf{Decrease and Conquer}
        \end{itemize}
    \end{itemize}
\end{frame}

\begin{frame}{Vorteile}{Von D\&C Algorithmen (Vgl. \cite{wiki:dac})}
    \begin{itemize}
        \item \textbf{Starke Lösungsstrategie}
        \begin{itemize}
            \item Hilft dabei, Lösungen für komplexe Probleme zu finden
            \item Solange man einen Weg findet, das Problem in kleinere Subprobleme zu teilen
        \end{itemize}
        \item \textbf{Algorithmeneffizienz}
        \begin{itemize}
            \item Der D\&C Ansatz hat oft effizientere Algorithmen für bekannte Probleme gefunden
            \item Zum Beispiel: Quicksort, Mergesort und FFT\footnote{Fast Fourier Transformation}
            \item Wenn sich ein Problem der Größe $n$ immer in $p$ Teilprobleme der Größe $~\frac{n}{p}$ teilen lässt, so ist die Komplexität von $O(n\log_p n)$
        \end{itemize}
    \end{itemize}
\end{frame}

\begin{frame}{Vorteile}{Von D\&C Algorithmen (Vgl. \cite{wiki:dac})}
    \begin{itemize}
        \item \textbf{Parallelisierung}
        \begin{itemize}
            \item Teilprobleme können oft parallel bearbeitet werden
            \item Dadurch teils erhebliche Zeitersparnis
        \end{itemize}
        \item \textbf{Speicherzugriff}
        \begin{itemize}
            \item D\&C Algorithmen können den Speicher meist effizienter (=schneller) nutzen
            \item Wenn die Teilprobleme klein genug sind können diese ggf. direkt im Prozessorcache berechnet werden
            \item Dieser ist im Vergleich zum RAM deutlich schneller durch höhere Taktraten und physische Nähe
        \end{itemize}
    \end{itemize}
\end{frame}

\begin{frame}{Herausforderungen}{Bei der Implementierung von D\&C Algorithmen (Vgl. \cite{wiki:dac})}
\begin{itemize}
    \item \textbf{Rekursion}
    \begin{itemize}
        \item Implementierung erfolgt in der Regel über rekursive Aufrufe
        \item Dies ist häufig komplexer in der Implementierung und dem Verständnis
    \end{itemize}
    \item \textbf{Aufruftiefe}
    \begin{itemize}
        \item Je nachdem wie oft der rekursive Aufruf erfolgt führt das zu Problemen
        \item Je nach Sprache und Compiler ist ggf. nur eine Rekursionstiefe möglich
        \item Dies kommt durch die ggf. beschränkte Größe des \textit{Call Stacks}
        \item Bei zu tiefer Rekursion kann es so zum \textit{Stack Overflow} kommen
    \end{itemize}
\end{itemize}
\end{frame}

\begin{frame}{Herausforderungen}{Bei der Implementierung von D\&C Algorithmen (Vgl. \cite{wiki:dac})}
    \begin{itemize}
        \item \textbf{Auswahl des "`trivialen Problems"'}
        \begin{itemize}
            \item Die Auswahl des direkt lösbaren Teilproblems ist nicht immer direkt ersichtlich
            \item In einigen Fällen ist ein Teilen bis zum kleinstmöglichen Teilproblem nicht sinnvoll
            \item Und effizienter ist es ein größeres Teilproblem direkt zu lösen
            \item Beispiel: Determinantenberechnung in Matrizen
        \end{itemize}
    \end{itemize}
\end{frame}