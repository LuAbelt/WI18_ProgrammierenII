\begin{frame}{Insertion Sort}{Grundprinzip (Siehe \cite{ottmann2017} S. 85ff)}
    \begin{itemize}
        \item Gegeben sein eine Liste mit $N$ Elementen
        \item Jedes Element wird nacheinander betrachtet
        \item Und an der korrekten Stelle der bereits betrachteten Elemente eingefügt
        \item Dadurch ergibt sich:
        \begin{itemize}
            \item Eine bereits sortierte Teilliste
            \item Eine Restliste mit den noch einzusortierenden Elementen
        \end{itemize}
    \end{itemize}
\end{frame}

\begin{frame}{Insertion Sort}{Praktisches Beispiel}
\end{frame}

\begin{frame}{Vor- und Nachteile (Siehe \cite{ottmann2017} S. 85ff)}
    \begin{itemize}
        \item Implementierung ist relativ simpel
        \item Jedoch viele Vergleiche und ggf. Verschiebungen nötig
        \item Komplexität beträgt hierfür $O(N^2)$
    \end{itemize}
\end{frame}

