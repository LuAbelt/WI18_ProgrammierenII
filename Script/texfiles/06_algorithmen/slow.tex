\begin{frame}{Slowsort}{Ein humoristischer Ansatz an Sortierungen}
    \begin{itemize}
        \item 1986 von Andrei Broder und Jorge Stolfi entwickelt
        \item Teil ihres Papers "`Pessimal Algorithms and Simplexity Analysis"
        \item Ziel war ein möglichst ineffizienten Algorithmus zu schaffen
        \begin{itemize}
            \item Ohne Nutzung von zufälligen Faktoren
            \item ...und ohne "`überflüssige"' Operationen einzubauen
        \end{itemize}
        \item Basiert auf dem \textbf{Multiply and Surrender} (Parodie auf Divide and Conquer) Prinzip
    \end{itemize}
\end{frame}

\begin{frame}{Slowsort}{Ablauf}
    \begin{itemize}
        \item Besteht im Grund aus zwei Schritten:
        \begin{itemize}
            \item 1. Finde das Maximum der Liste und platziere es am Ende
            \item 2. Sortiere die verbleibende Teilliste
        \end{itemize}
        \item Ineffizienz kommt durch die rekursive Umsetzung des ersten Schritts:
        \begin{itemize}
            \item 1.1 Finde (rekursiv) das Maximum der ersten Listenhälfte
            \item 1.2 Finde (rekursiv) das Maximum der zweiten Listenhälfte
            \item 1.3 Vergleiche die Maxima und tausche ggf.
        \end{itemize}
        \item Die untere Grenze der Komplexität lässt sich angeben mit $\Omega (n^{\frac{\log_2(n)}{2+\epsilon}})$
        \item Damit ist selbst der Best-Case schlechter als der Worst-Case von Bubble Sort
    \end{itemize}
\end{frame}