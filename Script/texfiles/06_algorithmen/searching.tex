\begin{frame}{Suchalgorithmen}{Motivation}
    \begin{itemize}
        \item Dienen dazu mit großen Datenmengen zu arbeiten
        \item Um bestimmte Informationen zu finden
        \item Beispiele (Digital und analog):
        \begin{itemize}
            \item Finden einer Übersetzung im Wörterbuch 
            \item Finden von Websites
            \item Suchen von bestimmten Buchabschnitten nach Thema (über Inhaltsverzeichnis oder Index)
        \end{itemize}
    \end{itemize}
\end{frame}

\begin{frame}{Allgemeine Aspekte}{Der Suche}
    \begin{itemize}
        \item Oft wird nach den \textit{Werten} für bestimmte \textit{Schlüssel} gesucht
        \item Die Suche in einer beliebigen Sammlung von Daten ist in der Regel nur schwer optimierbar
        \begin{itemize}
            \item Man muss jedes Element der Sammlung einzeln betrachten um ein bestimmtes Element zu finden
            \item Komplexität: $O(N)$
        \end{itemize}
        \item Aus diesen Gründen werden zum suchen teils spezielle Datenstrukturen verwendet:
        \begin{itemize}
            \item Symboltabellen
            \item Hashtables
            \item Suchbäume
        \end{itemize}
    \end{itemize}
\end{frame}

\begin{frame}{Allgemeine Aspekte}{Der Suche}
    \begin{itemize}
        \item Gemeinsamkeit der Suchstrukturen:
        \begin{itemize}
            \item Sind meist nach einem bestimmten Kriterium sortiert
        \end{itemize}
        \item Dadurch lassen sich die Strukturen deutlich einfacher durchsuchen
        \item Stichwort: \textbf{Binärsuche}
    \end{itemize}
\end{frame}