\begin{frame}{Exceptions}{Allgemein}
    \begin{itemize}[<+->]
        \item Signalisieren Fehler im entworfenen Programm
        \item Können auftreten:
        \begin{itemize}
            \item Zur Laufzeit (Unchecked Exceptions)
            \item Beim compilen (Checked Exceptions)
        \end{itemize}
        \item Beenden die Ausführung des Programms vorzeitig
        \begin{itemize}
            \item Wenn Sie nicht abgefangen werden
        \end{itemize}
    \end{itemize}
\end{frame}

\begin{frame}{Checked Exceptions}{}
    \begin{itemize}[<+->]
        \item Werden bereits durch den Compiler erkannt
        \item \textbf{Müssen} abgefangen werden
        \begin{itemize}
            \item Entweder direkt in der Methode
            \item Oder durch "`spezifizieren"' in der Methodensignatur
        \end{itemize}
        \item Beispiel:
        \begin{itemize}
            \item \texttt{FileNotFoundException}
        \end{itemize}
    \end{itemize}
\end{frame}

\begin{frame}[fragile]{Check Exception}{Beispiel}
\lstset{style=java}
\begin{lstlisting}
import java.io.*; 
  
class Main { 
    public static void main(String[] args) { 
        FileReader file = new FileReader("C:\\\\test\\\\a.txt"); 
        BufferedReader fileInput = new BufferedReader(file); 
          
        // Print first 3 lines of file "C:\\test\\a.txt" 
        for (int counter = 0; counter < 3; counter++)  
            System.out.println(fileInput.readLine()); 
          
        fileInput.close(); 
    } 
}
\end{lstlisting}
\end{frame}

\begin{frame}{Unchecked Exceptions}{Probleme zur Laufzeit}
    \begin{itemize}[<+->]
        \item Werden nicht durch den Compiler erkannt
        \begin{itemize}
            \item Müssen nicht abgefangen werden
            \item Beenden die Ausführung aber vorzeitig bei auftreten
        \end{itemize}
        \item Können somit zur Laufzeit des Programmes auftreten
        \item Sind meist Indikator für Programmierfehler
        \begin{itemize}
            \item Logikfehler
            \item Falsche Nutzung von APIs
        \end{itemize}
        \item Bekannte Vertreter:
        \begin{itemize}
            \item \texttt{NullPointerException}
            \item \texttt{ArrayIndexOutOfBoundsException}
            \item \texttt{IndexOutOfBoundsException}
            \item \texttt{NumberFormatException}
        \end{itemize}
    \end{itemize}
\end{frame}

\begin{frame}[fragile]{Checked Exceptions}{Beispiel}
\lstset{style=java}
\begin{lstlisting}
class Main { 
   public static void main(String args[]) { 
      int x = 0; 
      int y = 10; 
      int z = y/x; 
  } 
}
\end{lstlisting}
\end{frame}

\begin{frame}{Behandeln von Exceptions}{try-catch}
    \begin{itemize}[<+->]
        \item Fehler können über \texttt{try-catch} Konstrukte abgefangen werden
        \item Für verschiedene Fehler können unterschiedliche Abfangmethoden deklariert werden
        \item \texttt{finally} kann angegeben werden um Code auszuführen, nachdem der \texttt{try} Block bearbeitet wurde
        \begin{itemize}
            \item Unabhängig davon, ob in diesem ein Fehler auftrat
            \item Oder dieser behandelt wurde
            \item Kann zum Beispiel genutzt werden, um offene Streams zu schließen
        \end{itemize}
        \item Ein \texttt{catch} Block kann mehrere Fehler behandeln
    \end{itemize}
\end{frame}

\begin{frame}[fragile]{Behandeln von Exceptions}{Beispiel}
\lstset{style=java}
\begin{lstlisting}
public void readFile(String[] fNames, int index){
    try{
        String fName=fNames[index];
        FileReader file = new FileReader(fName); 
        BufferedReader fileInput = new BufferedReader(file); 
    } catch(IOException e){
    //Fehlerbehandlung für nicht gefundene Datei
    }  catch(ArrayIndexOutOfBoundsException e){
    //Fehlerbehandlung für invaliden Array Index
    }
    /*Weiterer Code*/
}
\end{lstlisting}
\end{frame}

\begin{frame}[fragile]{Behandeln von Exceptions}{Beispiel}
\lstset{style=java}
\begin{lstlisting}
public int calc(int[] data, int index, int divisor){
    try{
        return data[index]/divisor;
    } catch(ArrayIndexOutOfBoundsException|NumberFormatException e){
        return 0;
    }
}
\end{lstlisting}
\end{frame}

\begin{frame}{Behandeln von Exception}{Spezifizieren von Ausnahmen}
\begin{itemize}[<+->]
    \item Fehler müssen nicht direkt durch die aufrufende Methode behandelt werden
    \item Alternativ Nutzung des \texttt{throws}-Keywords möglich
    \begin{itemize}
        \item Gibt an, dass diese Methode den angegebenen Fehler auslösen kann
        \item Verantwortung zum behandeln des Fehlers liegt dann am Aufrufer der Methode
    \end{itemize}
    \item (Checked) Exceptions müssen spätestens in der \texttt{main()} Methode behandelt werden
\end{itemize}
\end{frame}

\begin{frame}[fragile]{Throws-Keyword}{Beispiel}
\lstset{style=java}
\begin{lstlisting}
public int getElement(int[] data, int i) throws ArrayIndexOutOfBoundsException{
    return data[i];
}
\end{lstlisting}
\end{frame}

\begin{frame}{Auslösen von Exceptions}{}
    \begin{itemize}
        \item Excpetions können auch "`manuell"' ausgelöst werden
        \item Um dem Anwender zu signalisieren, dass ein Fehler aufgetreten ist
        \item Häufig in Verbindung mit selbstdefinierten Fehlerklassen
        \item Auslösen über Nutzung des \texttt{throw} Keywords
        \begin{itemize}
            \item Ausgelöster Fehler muss das \texttt{Throwable} Interface implementieren
        \end{itemize}
    \end{itemize}
\end{frame}

\begin{frame}[fragile]{Throws-Keyword}{Beispiel}
\lstset{style=java}
\begin{lstlisting}
public void throwDemo(){
    throw new IOException();
}
\end{lstlisting}
\end{frame}

\begin{frame}{Custom Exceptions}{}
    \begin{itemize}
        \item Eigene Fehlertypen können über Klassen definiert werden, die das \texttt{Throwable} Interface implementieren
        \item Hierüber kann beispielsweise eine eigene Fehlermeldung definiert werden
        \item Bei eigenen Fehlern ist zu entscheiden, ob diese als Checked oder Unchecked Exceptions betrachtet werden sollen.
        \begin{itemize}
            \item Unchecked Exceptions erben von \texttt{RunTimeExceptions}
            \item Die Verwendung von unchecked Exceptions ist abzuwägen...
            \item ...Auch wenn es die Entwicklung "`vereinfacht"' (weil keine Compilerfehler auftreten)
        \end{itemize}
    \end{itemize}
\end{frame}