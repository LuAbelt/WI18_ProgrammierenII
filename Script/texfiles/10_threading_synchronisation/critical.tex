\begin{frame}{Gemeinsame Zugriffe}{Allgemein}
    \begin{itemize}
        \item "`Einfache"' Threads kommen sich nur selten in die Quere
        \item Wenn zur Bearbeitung nur lokale Variablen benötigt werden
        \item Jeder Thread hat seine eigenen Variablen und Callstack
        \item Probleme treten bei \textit{gemeinsam genutzten} Daten auf
        \item ...Oder Ressourcen
        \item Beispielsweise: Externe Daten(objekte), statische Variablen
    \end{itemize}
\end{frame}

\begin{frame}[fragile]{Statischer Zugriff}{Codebeispiel}
\lstset{style=java}
\begin{lstlisting}
class MyThread extends Thread{
static int result;
public void run(){ ... }
}
\end{lstlisting}
\end{frame}

\begin{frame}{Gemeinsame Zugriffe}{Weitere Beispiele}
    \begin{itemize}
        \item Parallele Zugriffe könnten auch bei globalen Datenstrukturen auftreten
        \item Wenn diese aus mehreren Threads aufgerufen und bearbeitet werden
        \item Lesen ist in der Regel das geringere Problem, schreibende Zugriffe sind aber problematisch
        \item Abstrakteres Beispiel: \textbf{Was würde passieren, wenn mehrere Prozesse parallel auf einen Drucker zugreifen, ohne dass dieser die Zugriffe einschränken würde?}
        \item Problematik liegt in der Umschaltung der Threads
        \item Es ist nicht absehbar, zu welchem Punkt die Bearbeitung umspringt
        \item Dadurch muss ein Mechanismus geschaffen werden, der sicherstellt, dass eine Ressource oder Datensatz nur durch einen Aktuer genutzt wird.
        \item Man spricht hierbei von \textbf{kritischen Abschnitten}
    \end{itemize}
\end{frame}

\begin{frame}{Kritische Abschnitte}{Allgemeines}
    \begin{itemize}
        \item Beschreiben Codeblöcke, in denen sichergestellt wird, dass nur ein Thread auf sie zugreift
        \item Man spricht auch von \textit{gegenseitigem Ausschluss} oder \textit{atomar}, wenn nur ein Thread in einem Programmteil arbeitet
        \item Lesende Zugriffe sind weniger kritisch
        \item Aber sobald Daten verändert werden, müssen kritische Abschnitte geschützt werden
        \item Wenn alle kritschen Abschnitte gesichert sind, spricht man von einer \textit{thread-sicheren} Implementierung (\textit{thread-safe})
        \item Immutable Objekte sind immer threadsicher (Da sie nicht verändert werden können)
    \end{itemize}
\end{frame}

\begin{frame}{Nicht kritische Abschnitte}{}
    \begin{itemize}
        \item Nicht kritische Abschnitte sind "`automatisch"' Threadsicher
        \item Deren parallele Ausführung hat keine unerwünschten Nebeneffekte
        \item Grundsätzlich alle Abschnitte in denen:
        \begin{itemize}
            \item Nur lesende Zugriffe erfolgen
            \item Keine Objekteigenschaften verändert werden
            \item Ausschließlich mit lokalen Variablen (Oder Parametervariablen) gearbeitet wird
        \end{itemize}
    \end{itemize}
\end{frame}

\begin{frame}{Paralleler Zugriff}{Beispiel}
    \begin{itemize}
        \item Folgendes Beispiel: Wir haben ein Objekt \texttt{p} vom Typ \texttt{Point}
        \item Thread \texttt{T1} möchte {p} mit $(1|1)$ belegen
        \item Thread \texttt{T2} möchte {p} (gleichzeitig) mit $(2|2)$ belegen
    \end{itemize}
    \begin{tabular}{|c|c|}
    \hline
    \textbf{Thread T1}&\textbf{Thread T2}\\\hline
    \texttt{p.x=1}&\texttt{p.x=2}\\\hline
    \texttt{p.y=1}&\texttt{p.y=2}\\\hline
    \end{tabular}
\end{frame}

\begin{frame}{Paralleler Zugriff}{Beispiel}
    \begin{itemize}
        \item Durch Umschaltung der Threads können die einzelnen Anweisungen "`durcheinander"' geraten:
    \end{itemize}
    \begin{tabular}{|c|c|c|}
    \hline
    \textbf{Thread T1}&\textbf{Thread T2}&\textbf{(x|y)}\\\hline
    \texttt{p.x=1}&&(1|0)\\\hline
    &\texttt{p.x=2}&(2|0)\\\hline
    &\texttt{p.y=2}&(2|2)\\\hline
    \texttt{p.y=1}&&(2|1)\\\hline
    \end{tabular}
\end{frame}

\begin{frame}{Paralleler Zugriff}{}
    \begin{itemize}
        \item Durch Umschaltung wird somit ungültiges Ergebnis erreicht
        \item Auch "`kleine"' Operationen sind nicht automatisch atomar
        \item Beispielsweise Benutzung von \texttt{i++}
        \item Da diese Operation (auf unterer Ebene) aus mehreren Teilen besteht:
        \begin{itemize}
            \item Hole den aktuellen Wert von i
            \item Deklariere den konstanten Wert 1
            \item Addiere i mit der Konstante
            \item Schreibe das neue i zurück in den vorgesehenen Speicherbereich
        \end{itemize}
    \end{itemize}
\end{frame}

\begin{frame}{Schützen}{Von kritischen Abschnitten}
    \begin{itemize}
        \item Wir müssen also einen Mechanismus finden um parallelen Zugriff zu vermeiden
        \item Java bietet hier zwei Konstrukte, mit denen wir uns beschäftigen:
        \begin{itemize}
            \item Das Verwenden von \textit{Lock} Objekten
            \item Die Nutzung des \texttt{synchronized} Keywords
        \end{itemize}
    \end{itemize}
\end{frame}