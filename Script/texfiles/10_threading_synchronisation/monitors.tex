\begin{frame}{Monitore}{Allgemein}
    \begin{itemize}
        \item Monitor wird nötig, wenn nur ein Thread in einen Block kommen soll
        \item In Java realisiert über \textit{Locks}
        \item Diese können einen Codeabschnitt "`abschließen"' und "`öffnen"'
        \item Das \texttt{Lock} Interface hat zwei wichtige Methoden:
        \begin{itemize}
            \item \texttt{lock()} -- Markiert den Beginn des kritischen Abschnitts. Wenn bereits (durch einen anderen Thread) ein kritischer Abschnitt betreten wurde, blockiert die Methode so lange bis dieser das Lock wieder freigibt
            \item \texttt{unlock()} -- Markiert das Verlassen des kritischen Abschnitts
        \end{itemize}
    \end{itemize}
\end{frame}

\begin{frame}{Lock Interface}{Wichtige Aspekte}
    \begin{itemize}
        \item Zu jedem \texttt{lock} muss auch ein \texttt{unlock} existieren
        \item Sonst können spätere Threads nie in den kritischen Abschnitt springen
        \item Das normale \texttt{lock} ignoriert den \texttt{interrupt} Aufruf des Threads
        \item Spezielle Methode \texttt{lockInterruptedly} realisiert abbrechbaren Aufruf
        \begin{itemize}
            \item \textbf{Achtung:} Insbesondere hier ist für den Fall, das unterbrochen wird (und eine Ausnahme ausgelöst wird), der \texttt{unlock} Aufruf wichtig!
            \item Wird am besten im \texttt{finally}-Block realisiert
        \end{itemize}
        \item Weitere Methode: \texttt{tryLock} -- Gibt sofort \texttt{true} oder \texttt{false} zurück und betritt entsprechend den kritischen Abschnitt
    \end{itemize}
\end{frame}

