\begin{frame}{Synchronized}{The old way}
    \begin{itemize}
        \item Über das \texttt{synchronized} Schlüsselwort lässt sich die gleiche Funktionalität wie mit \textit{Locks} erreichen
        \item Sind seit Java 1.0 Teil der Sprache
        \item Können jedoch direkt in Methodensignaturen verwendet werden
        \item Dadurch wird sichergestellt, dass eine Methode eines Objekts nur gleichzeitig durch einen Thread aufgerufen wird
        \item Auf verschiedenen Objekten kann die Methode weiterhin gleichzeitig aufgerufen werden
    \end{itemize}
\end{frame}

\begin{frame}{Synchronized}{In der Methodensignatur}
\lstset{style=java}
\begin{lstlisting}
synchronized void foo(){ i++; }
\end{lstlisting}
\end{frame}

\begin{frame}{Synchronized}{}
    \begin{itemize}
        \item 
    \end{itemize}
\end{frame}