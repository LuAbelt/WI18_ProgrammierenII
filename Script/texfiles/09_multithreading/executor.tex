\begin{frame}{Executor}{Grundlegendes (Vgl. \cite{ullenboom2014java} S. 197f)}
    \begin{itemize}
        \item Bisher haben wir die Threads immer direkt erzeugt
        \item Dies hat jedoch einige Nachteile:
        \begin{itemize}
            \item Bei erzeugen muss der Ausführbare Code (Meist als \texttt{Runnable}) übergeben werden $\rightarrow$ Erzeugen eines Threads und späteres festlegen des nebenläufigen Codes nicht möglich
            \item Jeder Thread kann nur einmal ausgeführt werden. Wiederholtes starten führt zu einem Fehler
            \item Thread wird immer sofort nach \texttt{start()} ausgeführt - Dies ist ggf. nicht zwangsweise gewollt
            \item Gesamtanzahl der Threads lässt sich nur schwer begrenzen
        \end{itemize}
        \item Hilfreich wäre eine Abstraktion, die die \textit{Runnables} von der technischen Umsetzung trennt
    \end{itemize}
\end{frame}

\begin{frame}{Executor}{Vorteile}
    \begin{itemize}
        \item Bietet eine Methode um beliebige \textit{Runnables} nebenläufig auszuführen:
        \begin{itemize}
            \item \texttt{submit(Runnable r)}
        \end{itemize}
        \item Java bietet verschiedene Standard-Executor:
        \begin{itemize}
            \item \texttt{ThreadPoolExecutor} -- Nutzt eine Sammlung von Threads (\textit{Thread-Pool}) die zur Ausführung wiederverwendet werden können
            \item \texttt{ScheduledThreadPoolExecutor} -- Erweiterung des normalen Pool Executors. Erlaubt die Ausführung zu einem bestimmten Zeitpunkt oder wiederholte Ausführung
        \end{itemize}
    \end{itemize}
\end{frame}

\begin{frame}{Executor}{Erzeugung}
    \begin{itemize}
        \item Werden am einfachsten erzeugt über die entsprechenden statischen Methoden:
        \begin{itemize}
            \item \texttt{static ExecutorService newCachedThreadPool()} -- Liefert einen Thread Pool mit wachsender Größe
            \item \texttt{static ExecutorService newFixedThreadPool(int nThreads)} -- Liefert einen Thread Pool mit fest definierter Größe
        \end{itemize}
    \end{itemize}
\end{frame}

\begin{frame}{Thread-Pools}{}
    \begin{itemize}
        \item Großer Vorteil der Executors: Thread Pools
        \item Das erzeugen neuer Threads ist sehr aufwändig
        \item Threads in Thread Pools können wiederverwendet werden
        \item Dadurch muss nicht für jedes \texttt{Runnable} ein neuer Thread erzeugt werden
    \end{itemize}
\end{frame}

\begin{frame}{Callable}{Rückgabewerte von Nebenläufigkeiten}
    \begin{itemize}
        \item \texttt{Runnable} kann keinen Wert zurückgeben -- Für Nebenläufige Berechnungen äußerst ungünstig
        \item Daher gibt es weiterhin das (generische) \texttt{Callable<V>}
        \item Auch wieder ein \textit{funktionales Interface}:
        \begin{itemize}
            \item \texttt{V call()} -- Äquivalent zur run Methode im Runnable, jedoch jetzt mit Rückgabewert
        \end{itemize}
        \item Auch wieder über Lambda Funktionen definierbar
    \end{itemize}
\end{frame}

\begin{frame}[fragile]{Callable}{Als eigene Klasse}
\lstset{style=java}
\begin{lstlisting}
public class MyCallable implements Callable<Integer>{
  @Override
  public void call(){
    return 42;
  }
}
//Alternativ:
Callable<Integer> c = ()->{return 42;};
\end{lstlisting}
\end{frame}

\begin{frame}{Future}{Ein Blick in die Zukunft}
    \begin{itemize}
        \item \texttt{Callable} Objekte werden auch über \texttt{submit} an den Executor gegeben
        \item Dieser gibt ein \texttt{Future<V>} Objekt zurück
        \item Darüber lassen sich verschiedene Aspekte prüfen:
        \begin{itemize}
            \item Ob die Berechnung abgeschlossen wurde
            \item Welches Ergebnis zurückgegeben wurde
            \item Ob die Bearbeitung abgebrochen wurde
        \end{itemize}
        \item Das \texttt{Future} Objekt kann die Berechnung auch abbrechen
    \end{itemize}
\end{frame}

\begin{frame}{Future}{Interface}
    \begin{itemize}
        \item \texttt{V get()} -- Gibt den Rückgabewert des \texttt{Callables} zurück
        \item \texttt{V get(long timeout, TimeUnit unit)} -- Gibt den Rückgabewert des \texttt{Callables} zurück mit einer maximalen Wartezeit
        \item \texttt{boolean isDone()} -- Überprüft, ob die Berechnung abgeschlossen ist
        \item \texttt{boolean cancel(boolean)} -- Bricht die Arbeit ab
        \item \texttt{boolean isCencelled()} -- Gibt an, ob die Arbeit unterbrochen wurde
    \end{itemize}
\end{frame}
