\begin{frame}{Unterbrechungen}{In Threads (Vgl. \cite{ullenboom2014java} S. 184f)}
    \begin{itemize}
        \item Threads bieten diverse Möglichkeiten zur Unterbrechung
        \item Simpelste: Die \texttt{sleep()} Methode
        \begin{itemize}
            \item Dieser wird (als \texttt{int}) die Zeit in Millisekunden übergeben, die pausiert werden soll
            \item Ist \textbf{nur} statisch vorhanden $\rightarrow$ Nur der eigene Thread lässt sich pausieren
            \item Kann eine \texttt{InterruptedException} auslösen, die über try-catch abgefangen werden muss
            \item Gleiche Funktionalität kann auch mit der \texttt{TimeUnit} Klasse erreicht werden $\rightarrow$ Dort eingängliche Definition der Zeit die gewartet wird
        \end{itemize}
    \end{itemize}
\end{frame}

\begin{frame}[fragile]{\texttt{sleep()}}{Codebeispiel}
\lstset{style=java}
\begin{lstlisting}
try {
  Thread.sleep(5000);
} catch(InterruptedException e) {}
//Alternativ:
try {
  TimeUnit.SECOND.sleep(5);
} catch(InterruptedException e) {}
\end{lstlisting}
\end{frame}

\begin{frame}{Verzicht}{Von Rechenzeit (Vgl. \cite{ullenboom2014java} S. 186f)} 
    \begin{itemize}
        \item Ein Thread kann von sich aus auf Rechenzeit verzichten
        \item Durch Aufruf von \texttt{Thread.yield()};
        \item Signalisiert dem Betriebssystem, dass auf weitere ggf. bereitgestellte Zeit verzichtet wird und direkt auf einen anderen Prozess geswitched werden kann
        \item \textbf{Nicht deterministisch} -- Wann der Thread wieder aufgenommen wird ist unklar
        \item Daher mit Bedacht zu verwenden
        \item Oft nur für sehr spezielle Anwendungsfälle wirklich praktikabel
    \end{itemize}
\end{frame}

\begin{frame}{Unterbrechen}{Über Interrupt (Vgl. \cite{ullenboom2014java} S. 189ff)}
    \begin{itemize}
        \item Häufig laufen in nebenläufigen Threads Anwendungen in Endlosschleifen um dauerhaft verfügbar zu sein
        \begin{itemize}
            \item Beispielsweise Server, die auf ankommende Verbindungen warten
            \item Durch die Endlosschleife würden diese auf natürliche Weise nicht beenden
        \end{itemize}
        \item Deshalb kann von außen an den Thread eine \textit{Anfrage auf Unterbrechung} gestellt werden
        \item Über die Methode \texttt{interrupt()}
        \item Anders als der Name ggf. vermuten lässt wird der Thread nicht direkt unterbrochen
        \item Es wird lediglich ein Flag im Thread gesetzt, die dieser eigenständig überprüfen muss
        \item Thread kann selbst über \texttt{isInterrupted()} prüfen, ob er unterbrochen wurde
    \end{itemize}
\end{frame}

\begin{frame}[fragile]{Interrupt}{Codebeispiel}
\lstset{style=java}
\begin{lstlisting}
Thread t = new Thread(()->{
  while(!isInterrupted()){
    System.out.println("Running...");
    try{
      Thread.sleep(500);
    } catch (InterruptedException e){
      interrupt();
      System.out.println("Interrupt caught while sleep()");
    }
  }
  System.out.println("Ende");
});
t.start();
Thread.sleep(2000);
t.interrupt();
\end{lstlisting}
\end{frame}

\begin{frame}{Daemons}{Speziell für Endlosthreads (Vgl. \cite{ullenboom2014java} S. 187f)}
    \begin{itemize}
        \item Für "`Endlosthreads"' im Hintergrund gibt es einen speziellen Modi
        \item Man kann den Thread als \textit{Daemon} markieren
        \item Das bedeutet: Der Thread wird solange ausgeführt, solange noch andere Threads (die keine Daemons sind) laufen
        \item Sollten nur noch Daemon Threads laufen, so werden diese alle beendet
        \item Markierung als Daemon über die Methode \texttt{setDaemon(boolean)}
    \end{itemize}
\end{frame}

\begin{frame}{Rendezvous}{Zusammenführen von Threads (Vgl. \cite{ullenboom2014java} S. 194f)}
    \begin{itemize}
        \item Threads werden unter anderem genutzt, um nebenläufige Berechnungen durchzuführen
        \item Die Ergebnisse werden an späterer Stelle benötigt
        \item Zum Verwendungszeitpunkt muss sichergestellt sein, dass die Berechnung abgeschlossen ist
        \item Dafür wird die Methode \texttt{join()} genutzt
        \item Diese blockiert die Ausführung so lange, bis der entsprechende Thread beendet ist (Die \texttt{run()} Methode beendet wurde)
    \end{itemize}
\end{frame}

\begin{frame}[fragile]{\texttt{join()}}{Codebeispiel}
\lstset{style=java}
\begin{lstlisting}
public class JoinThread extends Thread{
  public int result =0;
  @Override
  public void run(){
    result=1;
  }
}
//Anwendung
JoinThread t = new JoinThread();
t.start();
//t.join();
System.out.println(t.result);
\end{lstlisting}
\end{frame}

\begin{frame}{Rendezvous}{Überladungen}
    \begin{itemize}
        \item Über verschiedene Überladungen lässt sich eine maximale Wartezeit definieren:
        \begin{itemize}
            \item \texttt{join(long millis)} -- Definiert eine Wartezeit in Millisekunden
            \item \texttt{join(long millis, long nanos)} -- Definiert eine Wartezeit in Milli- und Nanosekunden
        \end{itemize}
        \item Nach \texttt{join} Aufruf mit Timeout kann über \texttt{isAlive()} geprüft werden, ob die Berechnung wirklich abgeschlossen wurde
    \end{itemize}
\end{frame}