\begin{frame}{Thread Klasse}{Überblick (Vgl. \cite{ullenboom2018java} S. 948f)}
    \begin{itemize}
        \item Neue Threads werden über die \texttt{Thread} Klasse erzeugt und verwaltet
        \item In der Regel greifen diese direkt auf die Thread Funktionen des Betriebssystems zu
        \begin{itemize}
            \item "`Native Threads"'
        \end{itemize}
        \item JVM Spezifikation schreibt jedoch nicht vor, ob im Hintergrund native Threads genutzt werden oder nicht
        \item Java garantiert nur, dass die Ausführung der Thread Implementierung korrekt und konsistent funktioniert
    \end{itemize}
\end{frame}

\begin{frame}{Thread Klasse}{Grundlegende Methoden (Vgl \cite{ullenboom2014java} S. 177ff)}
    \begin{itemize}
        \item \texttt{Thread} verfügt über zwei grundlegende Methoden:
        \begin{itemize}
            \item \texttt{run()} -- Diese Methode wird nebenläufig ausgeführt
            \item \texttt{start()} -- Startet den Thread und führt die \texttt{run()} Methode nebenläufig aus
        \end{itemize}
        \item \textbf{Achtung:} Zur nebenläufigen Ausführung muss \texttt{start()} aufgerufen werden. Wird \texttt{run()} direkt aufgerufen, dann wird der Code zwar auch ausgeführt, jedoch nicht nebenläufig!
        \item Eine Möglichkeit um nebenläufigen Code auszuführen ist es, eine eigene Unterklasse von \texttt{Thread} zu bilden, die die \texttt{run()} Methode überschreibt
    \end{itemize}
\end{frame}

\begin{frame}[fragile]{Thread}{Eigene Unterklasse (Vgl. \cite{ullenboom2014java} S. 180)}
\lstset{style=java}
\begin{lstlisting}
public class MyThread extends Thread{
  @Override
  public void run(){
    for(int i=0;i<100;i++){
      System.out.println(i);
    }
  }
}
\end{lstlisting}
\end{frame}

\begin{frame}{Thread Klasse}{Eigene Unterklasse (Vgl. \cite{ullenboom2014java} S. 182)}
    \begin{itemize}
        \item Bilden einer Subklasse von Thread ist in der Regel nicht sinnvoll
        \item Da in der Regel nur das \texttt{run} Verhalten definiert werden soll
        \item Die eigentliche Art und Weise der Nebenläufigen Bearbeitung wird nicht verändert
        \item Grundsatz beim Bilden von Unterklassen: In der Regel nur, wenn die grundlegende Funktionsweise spezifiziert werden soll
        \item Weiterer Nachteil: Vererbung ist schon "`aufgebraucht"'
        \item Daher bietet Java für Nebenläufigkeit auch ein Interface
    \end{itemize}
\end{frame}

\begin{frame}{Runnable Interface}{Grundlegendes(Vgl. \cite{ullenboom2014java} S. 177f)}
    \begin{itemize}
        \item Das \texttt{Runnable} Interface kann auch genutzt werden um nebenläufiges Verhalten abzubilden
        \item \texttt{Runnable} ist ein \textit{funktionales Interface}
        \begin{itemize}
            \item \texttt{void run()} -- Spezifiziert den nebenläufig auszuführenden Code
        \end{itemize}
        \item Für \texttt{Thread} gibt es einen Konstruktor, der als Parameter ein \texttt{Runnable} akzeptiert:
        \begin{itemize}
            \item \texttt{Thread(Runnable r)} -- Das \texttt{Runnable} Objekt definiert, welcher Code nebenläufig ausgeführt werden soll
            \item Thread muss weiterhin über \texttt{start()} gestartet werden
        \end{itemize}
        \item Da es sich bei \texttt{Runnable} um ein funktionales Interface handelt können auch Lambda Expressions verwendet werden
    \end{itemize}
\end{frame}

\begin{frame}[fragile]{Runnable}{Als eigene Klasse}
\lstset{style=java}
\begin{lstlisting}
public class MyRunnable implements Runnable{
  @Override
  public void run(){
    for(int i=0;i<100;i++){
      System.out.println(i);
    }
  }
}
//Verwendung:
Thread t = new Thread(new MyRunnable());
t.start();
\end{lstlisting}
\end{frame}

\begin{frame}[fragile]{Runnable}{Als Lambda Expression}
\lstset{style=java}
\begin{lstlisting}
Thread t = new Thread( () -> {
    for(int i=0;i<100;i++){
      System.out.println(i);
    }
  });
t.start()
\end{lstlisting}
\end{frame}

\begin{frame}{Runnable}{Aufgabe}
    \begin{alertblock}{}
    Implementiert zwei \texttt{Runnable} Klassen, die (verschiedene) Daten auf der Konsole ausgeben. Erzeugt dann in eurer \texttt{main()} Methode jeweils einen Thread zum Ausführen der Runnables und startet diese direkt nacheinander. 
    
    Was für eine Ausgabe erwartet ihr auf der Konsole?
    
    Inwiefern unterscheidet sich die tatsächliche Ausgabe von Eurer Erwartung?
    
    Beobachtet das Verhalten der Ausgaben bei mehrfacher Ausführung des Programms.
    
    \textit{Hinweis:} Für eine bessere Beobachtung ist es empfehlenswert, die Daten mehrfach über eine Schleife (>100 Iterationen) ausgeben zu lassen
    \end{alertblock}
\end{frame}