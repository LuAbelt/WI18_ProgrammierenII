\begin{frame}{Eigenschaften}{Von Threads (Vgl. \cite{ullenboom2014java} S. 182f)}
    \begin{itemize}
        \item Beispiele für Eigenschaften von Threads:
        \begin{itemize}
            \item Priorität
            \item Zustand
            \item Name
        \end{itemize}
        \item Name lässt sich festlegen im Konstruktor:
        \begin{itemize}
            \item \texttt{Thread(Runnable r, String name)}
            \item \texttt{Thread(String name)}
        \end{itemize}
        \item Oder über die entsprechende Funktion:
        \begin{itemize}
            \item \texttt{void setName(String name)}
        \end{itemize}
    \end{itemize}
\end{frame}

\begin{frame}{Eigenschaften}{Zugriff auf die Eigenschaften (Vgl. \cite{ullenboom2014java} S. 183)}
    \begin{itemize}
        \item Analog gibt es auch eine \texttt{getName()} Funktion
        \item \textit{Erinnerung:} Die Methoden gehören zu der \texttt{Thread} Klasse!
        \item Nicht direkt in \texttt{Runnable} abrufbar!
        \item Jedoch lässt sich der Thread in dem gerade gearbeitet wird zurückgeben:
        \begin{itemize}
            \item \texttt{static Thread currentThread()}
            \item Aufruf: \texttt{Thread t = Thread.currentThread()}
        \end{itemize}
        \item Wenn der aktuelle Thread mehrfach benutzt wird (z.B. in einer Schleife) sollte dieser vor der Schleife abgerufen und zwischengespeichert werden
        \begin{itemize}
            \item \texttt{currentThread()} Aufruf ist "`teuer"'
        \end{itemize}
    \end{itemize}
\end{frame}

\begin{frame}{Zustände}{Von Threads (Vgl. \cite{ullenboom2014java} S. 183f)}
    \begin{itemize}
        \item Lebenszyklus des Threads besteht aus verschiedenen Phasen:
        \begin{itemize}
            \item \textit{Nicht erzeugt}: Der Thread wurde schon definiert (über \texttt{new Thread(...)}), aber noch nicht gestartet
            \item \textit{Laufend}: Der Thread wurde gestartet (über die \texttt{start()} Methode)und wird aktuell durch den Prozessor bearbeitet
            \item \textit{Nicht Laufend}: Der Thread wurde gestartet (über die \texttt{start()} Methode), wird aber aktuell nicht bearbeitet
            \item \textit{Wartend}: Der Thread wartet auf ein bestimmtes Ereignis (Abschluss einer anderen Operation, Freiwerden einer Ressource...)
            \item \textit{Beendet}: Der Thread wurde fertig bearbeitet
        \end{itemize}
        \item Java bildet das über die \texttt{Thread.State} Enumeration ab
        \item Abrufbar über die \texttt{getState()} Methode
        \item Methode \texttt{isAlive()} gibt zurück, ob der Thread noch arbeitet oder beendet ist
    \end{itemize}
\end{frame}

\begin{frame}{Zustände}{In \texttt{Thread.State} (Siehe \cite{ullenboom2014java} S. 184)}
    \begin{tabular}{|c|p{9cm}|}
    \hline
    \textbf{Zustand}&\textbf{Erklärung}\\
    \hline
    \hline
    \texttt{NEW} & Thread ist erzeugt, aber noch nicht gestartet\\\hline
    \texttt{RUNNABLE} & Thread wurde gestartet und läuft in der JVM\\\hline
    \texttt{BLOCKED} & Thread wartet auf das freiwerden eines bestimmten Locks um beispielsweise einen synchronisierten Codeabschnitt zu betreten\\\hline
    \texttt{WAITING} & Wartet etwa auf ein \texttt{notify()}\\\hline
    \texttt{TIMED\_WAITING} & Zeitgesteuertes Warten beispielsweise durch \texttt{sleep()}\\\hline
    \texttt{TERMINATED} & Thread wurde beendet\\\hline
    \end{tabular}
\end{frame}