%%%%%%%%%%%%%%%%%%%%%%%%%%%%%%%%%%%%%%%%%%%%%%%%%%%%%%%%%%%%%%%%%%%%%%%%%%%%%%%%%%%%%%%%%%
%%
%% Description:		This is an example presentation using the beamerthemedhbw
%%
%%					The beamerthemedhbw is based on jacksbeamertheme
%%					(https://github.com/JacknJo/jacksbeamertheme)
%%
%% Author:			Hannes Bartle																				
%% 					DHBW Ravensburg Campus Friedrichshafen		
%%					September 2016	
%% 
%% The beamerthemedhbw is free software: you can redistribute it and/or modify
%% it under the terms of the GNU General Public License as published by
%% the Free Software Foundation, either version 3 of the License, or
%% (at your option) any later version.
%% 
%% The beamerthemedhbw is distributed in the hope that it will be useful,
%% but WITHOUT ANY WARRANTY; without even the implied warranty of
%% MERCHANTABILITY or FITNESS FOR A PARTICULAR PURPOSE.  See the
%% GNU General Public License for more details.
%% 
%% You should have received a copy of the GNU General Public License
%% along with the beamerthemedhbw.  If not, see <http://www.gnu.org/licenses/>.
%% 
%% 
%%%%%%%%%%%%%%%%%%%%%%%%%%%%%%%%%%%%%%%%%%%%%%%%%%%%%%%%%%%%%%%%%%%%%%%%%%%%%%%%%%%%%%%%%%


\documentclass[	12pt, 				
				t,					
				aspectratio=169,
				%handout-PLACEHOLDER
				]{beamer}

\usepackage{dhbwstyle}

\title{Nebenläufigkeit}

\begin{document}
	
	\begin{frame}[noframenumbering]
		\titlepage
	\end{frame}

	\begin{frame}{Inhalt}
		\tableofcontents
	\end{frame}
    
    \outlineFrame{Allgemeines}
    \begin{frame}{Multithreading}{Allgemeines (Vgl. \cite{ullenboom2018java} S. 948)}
    \begin{itemize}
        \item Moderne Betriebssysteme unterstützen \textit{Multitasking}
        \item Bedeutet: Mehrere Programme können gleichzeitig laufen
        \item Man spricht in der Regel von \textit{Nebenläufigkeit}
        \item Wie diese erreicht wird steuert das Betriebssystem (und ggf. Hardware)
        \item Auf Mehrkernprozessoren können "`echt parallel"' arbeiten
        \item Auf Einkernsystemen wird eine Parallelität "`simulisert"' (\textit{Quasiparallelität})
    \end{itemize}
\end{frame}

\begin{frame}{Multithreading}{Technische Umsetzung (Vgl. \cite{ullenboom2018java} S. 948)}
    \begin{itemize}
        \item Jeder Prozessorkern kann (in der Regel) zu einem Zeitpunkt einen Prozess bearbeiten
        \item In der Regel gibt es deutlich mehr laufende Prozesse als Kerne
        \item Lösung: Aktiver Prozess wird auf den Kernen hochfrequent (im Millisekundenbereich) umgeschalten
        \item Umschaltung erfolgt durch den \textit{Scheduler}
        \item Zur Umschaltung der Prozesse gibt es diverse Strategien mit diversen Parametern:
        \begin{itemize}
            \item Priorität
            \item Bearbeitungsdauer
            \item "`Fail-Count"' -- Wie oft wurde der Prozess schon versucht zu bearbeiten
        \end{itemize}
    \end{itemize}
\end{frame}

\begin{frame}{Prozesse}{Grundlegende Eigenschaften (Vgl. \cite{ullenboom2018java} S. 948)}
    \begin{itemize}
        \item Jeder Prozess besteht im Grunde aus:
        \begin{itemize}
            \item Dem auszuführenden Programmcode
            \item Den dazugehörigen Daten
            \item Einem \textit{eigenen} (isolierten) Speicherbereich
            \item Ggf. Verwendete Ressourcen wie Dateien oder Laufwerke
        \end{itemize}
        \item Durch die Trennung des Speicherbereichs können Prozesse nicht auf die Daten anderer Prozesse zugreifen!
        \item Ist doch ein Datenaustausch zwischen Prozessen erforderlich, ist ein spezielle \textit{Shared Memory} Bereich notwendig
        \item Prozesse können aus mehreren parallelen Threads bestehen $\rightarrow$ Diese können die gleichen Ressourcen nutzen
    \end{itemize}
\end{frame}

\begin{frame}{Nebenläufigkeit}{Geschwindigkeitsgewinn (Vgl. \cite{ullenboom2018java} S. 949ff)}
    \begin{itemize}
        \item Nebenläufigkeit führt in der Regel zu Geschwindigkeitsgewinn
        \item In Mehrkernsystemen sowieso...
        \item ...aber auch in Einkernsystemen
        \item Beispiel: Software zur Erstellung von Datenbank-Reports:
        \begin{itemize}
            \item Baue ein Fenster auf
            \item Öffnen der Datenbank vom Server, lesen der Datensätze
            \item Analyse der Daten, Visualisierung des Fortschritts
            \item Datei öffnen, Analyseergebnisse in Datei schreiben
        \end{itemize}
    \end{itemize}
\end{frame}

\begin{frame}{Nebenläufigkeit}{Beispiel für Geschwindigkeitsgewinn (Vgl. \cite{ullenboom2018java} S. 949ff)}
    \begin{itemize}
        \item Betrachten wir einmal die parallelisierbaren Abschnitte:
        \begin{itemize}
            \item Öffnen von Fenster und Datenbank können parallel geschehen
            \item Lesen neuer Datensätze und Analyse alter Datensätze kann parallel erfolgen
            \item Analyse neuer Datensätze und schreiben von alten analysierten Daten kann gleichzeitig abgearbeitet werden
        \end{itemize}
        \item Hier auch auf einem Einprozessorsystem großer Leistungsgewinn
        \item Da die parallelen Prozesse verschiedene \textit{Ressourcen} belasten
    \end{itemize}
\end{frame}

\begin{frame}{Nebenläufigkeit}{Beispiel für Geschwindigkeitsgewinn (Vgl. \cite{ullenboom2018java} S. 949ff)}
    \begin{itemize}
        \item Während auf das Fertigstellen einer Ressource gewartet wird, können Aufgaben bearbeitet werden die andere Ressourcen benötigen:
        \begin{itemize}
            \item Während der Prozessor ausgelastet ist die GUI zu erstellen kann eine Datei auf der Festplatte geöffnet werden $\rightarrow$ Dateioperationen benötigen wenig Prozessorleistung, eher durch Festplattengeschwindigkeit begrenzt
            \item Während Daten z.B. aus einer Datenbank abgerufen werden wird hauptsächlich die Netzwerkressource belastet $\rightarrow$ Prozessorleistung kann ggf. anders genutzt werden
            \item Parallel zu einer Prozessorlastigen Analyse können bereits analysierte Daten in eine Datei geschrieben werden
        \end{itemize}
        \item Kurz gesagt: Wir nutzen "`Wartezeiten"' von langsamen Operationen zu unserem Vorteil
    \end{itemize}
\end{frame}

\begin{frame}{Nebenläufigkeit}{Fazit (Siehe \cite{ullenboom2018java} S. 951)}
    \begin{itemize}
        \item Nebenläufigkeit muss gut geplant werden
        \item Insbesondere für Einkernsysteme
        \item Geschwindigkeitsgewinn nur vorhanden, wenn die parallelen Aktivitäten unterschiedliche Ressourcen nutzen
        \item Durch Nebenläufigkeit entsteht auch ggf. zusätzlicher Overhead für Synchronisation
        \item Zum Beispiel, wenn auf ein Teilergebnis gewartet werden muss
        \item Hier muss insbesondere auf konkurrierende Zugriffe und gegenseitige Wartebedingungen geachtet werden, um \textit{Deadlocks} zu vermeiden
    \end{itemize}
\end{frame}


    
    \outlineFrame{Thread Interface}
    \outlineSubframe{Runnable}
    \outlineSubframe{Zustände von Threads}
    \outlineSubframe{Unterbrechen und Beenden}
    
    \outlineFrame{Executors}
    \outlineSubframe{Callable}
    \outlineSubframe{Future}
    
    
    \printbibliographyframe
    
	\section*{Kontakt}
	\begin{frame}{Kontakt}{}
	\begin{itemize}
		\item E-Mail: \href{mailto:lukas.abelt@airbus.com}{lukas.abelt@airbus.com}
		\item GitHub: \url{https://www.github.com/LuAbelt}
		\item GitLab: \url{https://www.gitlab.com/LuAbelt}
		\item Telefon(Firma): 07545 - 8 8895
		\item Telegram: LuAbelt
	\end{itemize}
\end{frame}
	

\end{document}