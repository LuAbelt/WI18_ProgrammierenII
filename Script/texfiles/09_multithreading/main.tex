%%%%%%%%%%%%%%%%%%%%%%%%%%%%%%%%%%%%%%%%%%%%%%%%%%%%%%%%%%%%%%%%%%%%%%%%%%%%%%%%%%%%%%%%%%
%%
%% Description:		This is an example presentation using the beamerthemedhbw
%%
%%					The beamerthemedhbw is based on jacksbeamertheme
%%					(https://github.com/JacknJo/jacksbeamertheme)
%%
%% Author:			Hannes Bartle																				
%% 					DHBW Ravensburg Campus Friedrichshafen		
%%					September 2016	
%% 
%% The beamerthemedhbw is free software: you can redistribute it and/or modify
%% it under the terms of the GNU General Public License as published by
%% the Free Software Foundation, either version 3 of the License, or
%% (at your option) any later version.
%% 
%% The beamerthemedhbw is distributed in the hope that it will be useful,
%% but WITHOUT ANY WARRANTY; without even the implied warranty of
%% MERCHANTABILITY or FITNESS FOR A PARTICULAR PURPOSE.  See the
%% GNU General Public License for more details.
%% 
%% You should have received a copy of the GNU General Public License
%% along with the beamerthemedhbw.  If not, see <http://www.gnu.org/licenses/>.
%% 
%% 
%%%%%%%%%%%%%%%%%%%%%%%%%%%%%%%%%%%%%%%%%%%%%%%%%%%%%%%%%%%%%%%%%%%%%%%%%%%%%%%%%%%%%%%%%%


\documentclass[	12pt, 				
				t,					
				aspectratio=169,
				%handout-PLACEHOLDER
				]{beamer}

\usepackage{dhbwstyle}

\title{Nebenläufigkeit}

\begin{document}
	
	\begin{frame}[noframenumbering]
		\titlepage
	\end{frame}

	\begin{frame}{Inhalt}
		\tableofcontents
	\end{frame}
    
    \outlineFrame{Allgemeines}
    \begin{frame}{Multithreading}{Allgemeines (Vgl. \cite{ullenboom2018java} S. 948)}
    \begin{itemize}
        \item Moderne Betriebssysteme unterstützen \textit{Multitasking}
        \item Bedeutet: Mehrere Programme können gleichzeitig laufen
        \item Man spricht in der Regel von \textit{Nebenläufigkeit}
        \item Wie diese erreicht wird steuert das Betriebssystem (und ggf. Hardware)
        \item Auf Mehrkernprozessoren können "`echt parallel"' arbeiten
        \item Auf Einkernsystemen wird eine Parallelität "`simulisert"' (\textit{Quasiparallelität})
    \end{itemize}
\end{frame}

\begin{frame}{Multithreading}{Technische Umsetzung (Vgl. \cite{ullenboom2018java} S. 948)}
    \begin{itemize}
        \item Jeder Prozessorkern kann (in der Regel) zu einem Zeitpunkt einen Prozess bearbeiten
        \item In der Regel gibt es deutlich mehr laufende Prozesse als Kerne
        \item Lösung: Aktiver Prozess wird auf den Kernen hochfrequent (im Millisekundenbereich) umgeschalten
        \item Umschaltung erfolgt durch den \textit{Scheduler}
        \item Zur Umschaltung der Prozesse gibt es diverse Strategien mit diversen Parametern:
        \begin{itemize}
            \item Priorität
            \item Bearbeitungsdauer
            \item "`Fail-Count"' -- Wie oft wurde der Prozess schon versucht zu bearbeiten
        \end{itemize}
    \end{itemize}
\end{frame}

\begin{frame}{Prozesse}{Grundlegende Eigenschaften (Vgl. \cite{ullenboom2018java} S. 948)}
    \begin{itemize}
        \item Jeder Prozess besteht im Grunde aus:
        \begin{itemize}
            \item Dem auszuführenden Programmcode
            \item Den dazugehörigen Daten
            \item Einem \textit{eigenen} (isolierten) Speicherbereich
            \item Ggf. Verwendete Ressourcen wie Dateien oder Laufwerke
        \end{itemize}
        \item Durch die Trennung des Speicherbereichs können Prozesse nicht auf die Daten anderer Prozesse zugreifen!
        \item Ist doch ein Datenaustausch zwischen Prozessen erforderlich, ist ein spezielle \textit{Shared Memory} Bereich notwendig
        \item Prozesse können aus mehreren parallelen Threads bestehen $\rightarrow$ Diese können die gleichen Ressourcen nutzen
    \end{itemize}
\end{frame}

\begin{frame}{Nebenläufigkeit}{Geschwindigkeitsgewinn (Vgl. \cite{ullenboom2018java} S. 949ff)}
    \begin{itemize}
        \item Nebenläufigkeit führt in der Regel zu Geschwindigkeitsgewinn
        \item In Mehrkernsystemen sowieso...
        \item ...aber auch in Einkernsystemen
        \item Beispiel: Software zur Erstellung von Datenbank-Reports:
        \begin{itemize}
            \item Baue ein Fenster auf
            \item Öffnen der Datenbank vom Server, lesen der Datensätze
            \item Analyse der Daten, Visualisierung des Fortschritts
            \item Datei öffnen, Analyseergebnisse in Datei schreiben
        \end{itemize}
    \end{itemize}
\end{frame}

\begin{frame}{Nebenläufigkeit}{Beispiel für Geschwindigkeitsgewinn (Vgl. \cite{ullenboom2018java} S. 949ff)}
    \begin{itemize}
        \item Betrachten wir einmal die parallelisierbaren Abschnitte:
        \begin{itemize}
            \item Öffnen von Fenster und Datenbank können parallel geschehen
            \item Lesen neuer Datensätze und Analyse alter Datensätze kann parallel erfolgen
            \item Analyse neuer Datensätze und schreiben von alten analysierten Daten kann gleichzeitig abgearbeitet werden
        \end{itemize}
        \item Hier auch auf einem Einprozessorsystem großer Leistungsgewinn
        \item Da die parallelen Prozesse verschiedene \textit{Ressourcen} belasten
    \end{itemize}
\end{frame}

\begin{frame}{Nebenläufigkeit}{Beispiel für Geschwindigkeitsgewinn (Vgl. \cite{ullenboom2018java} S. 949ff)}
    \begin{itemize}
        \item Während auf das Fertigstellen einer Ressource gewartet wird, können Aufgaben bearbeitet werden die andere Ressourcen benötigen:
        \begin{itemize}
            \item Während der Prozessor ausgelastet ist die GUI zu erstellen kann eine Datei auf der Festplatte geöffnet werden $\rightarrow$ Dateioperationen benötigen wenig Prozessorleistung, eher durch Festplattengeschwindigkeit begrenzt
            \item Während Daten z.B. aus einer Datenbank abgerufen werden wird hauptsächlich die Netzwerkressource belastet $\rightarrow$ Prozessorleistung kann ggf. anders genutzt werden
            \item Parallel zu einer Prozessorlastigen Analyse können bereits analysierte Daten in eine Datei geschrieben werden
        \end{itemize}
        \item Kurz gesagt: Wir nutzen "`Wartezeiten"' von langsamen Operationen zu unserem Vorteil
    \end{itemize}
\end{frame}

\begin{frame}{Nebenläufigkeit}{Fazit (Siehe \cite{ullenboom2018java} S. 951)}
    \begin{itemize}
        \item Nebenläufigkeit muss gut geplant werden
        \item Insbesondere für Einkernsysteme
        \item Geschwindigkeitsgewinn nur vorhanden, wenn die parallelen Aktivitäten unterschiedliche Ressourcen nutzen
        \item Durch Nebenläufigkeit entsteht auch ggf. zusätzlicher Overhead für Synchronisation
        \item Zum Beispiel, wenn auf ein Teilergebnis gewartet werden muss
        \item Hier muss insbesondere auf konkurrierende Zugriffe und gegenseitige Wartebedingungen geachtet werden, um \textit{Deadlocks} zu vermeiden
    \end{itemize}
\end{frame}


    
    \outlineFrame{Threads}
    
    \outlineSubframe{Thread\&Runnable}
    \begin{frame}{Thread Klasse}{Überblick (Vgl. \cite{ullenboom2018java} S. 948f)}
    \begin{itemize}
        \item Neue Threads werden über die \texttt{Thread} Klasse erzeugt und verwaltet
        \item In der Regel greifen diese direkt auf die Thread Funktionen des Betriebssystems zu
        \begin{itemize}
            \item "`Native Threads"'
        \end{itemize}
        \item JVM Spezifikation schreibt jedoch nicht vor, ob im Hintergrund native Threads genutzt werden oder nicht
        \item Java garantiert nur, dass die Ausführung der Thread Implementierung korrekt und konsistent funktioniert
    \end{itemize}
\end{frame}

\begin{frame}{Thread Klasse}{Grundlegende Methoden (Vgl \cite{ullenboom2014java} S. 177ff)}
    \begin{itemize}
        \item \texttt{Thread} verfügt über zwei grundlegende Methoden:
        \begin{itemize}
            \item \texttt{run()} -- Diese Methode wird nebenläufig ausgeführt
            \item \texttt{start()} -- Startet den Thread und führt die \texttt{run()} Methode nebenläufig aus
        \end{itemize}
        \item \textbf{Achtung:} Zur nebenläufigen Ausführung muss \texttt{start()} aufgerufen werden. Wird \texttt{run()} direkt aufgerufen, dann wird der Code zwar auch ausgeführt, jedoch nicht nebenläufig!
        \item Eine Möglichkeit um nebenläufigen Code auszuführen ist es, eine eigene Unterklasse von \texttt{Thread} zu bilden, die die \texttt{run()} Methode überschreibt
    \end{itemize}
\end{frame}

\begin{frame}[fragile]{Thread}{Eigene Unterklasse (Vgl. \cite{ullenboom2014java} S. 180)}
\lstset{style=java}
\begin{lstlisting}
public class MyThread extends Thread{
  @Override
  public void run(){
    for(int i=0;i<100;i++){
      System.out.println(i);
    }
  }
}
\end{lstlisting}
\end{frame}

\begin{frame}{Thread Klasse}{Eigene Unterklasse (Vgl. \cite{ullenboom2014java} S. 182)}
    \begin{itemize}
        \item Bilden einer Subklasse von Thread ist in der Regel nicht sinnvoll
        \item Da in der Regel nur das \texttt{run} Verhalten definiert werden soll
        \item Die eigentliche Art und Weise der Nebenläufigen Bearbeitung wird nicht verändert
        \item Grundsatz beim Bilden von Unterklassen: In der Regel nur, wenn die grundlegende Funktionsweise spezifiziert werden soll
        \item Weiterer Nachteil: Vererbung ist schon "`aufgebraucht"'
        \item Daher bietet Java für Nebenläufigkeit auch ein Interface
    \end{itemize}
\end{frame}

\begin{frame}{Runnable Interface}{Grundlegendes(Vgl. \cite{ullenboom2014java} S. 177f)}
    \begin{itemize}
        \item Das \texttt{Runnable} Interface kann auch genutzt werden um nebenläufiges Verhalten abzubilden
        \item \texttt{Runnable} ist ein \textit{funktionales Interface}
        \begin{itemize}
            \item \texttt{void run()} -- Spezifiziert den nebenläufig auszuführenden Code
        \end{itemize}
        \item Für \texttt{Thread} gibt es einen Konstruktor, der als Parameter ein \texttt{Runnable} akzeptiert:
        \begin{itemize}
            \item \texttt{Thread(Runnable r)} -- Das \texttt{Runnable} Objekt definiert, welcher Code nebenläufig ausgeführt werden soll
            \item Thread muss weiterhin über \texttt{start()} gestartet werden
        \end{itemize}
        \item Da es sich bei \texttt{Runnable} um ein funktionales Interface handelt können auch Lambda Expressions verwendet werden
    \end{itemize}
\end{frame}

\begin{frame}[fragile]{Runnable}{Als eigene Klasse}
\lstset{style=java}
\begin{lstlisting}
public class MyRunnable implements Runnable{
  @Override
  public void run(){
    for(int i=0;i<100;i++){
      System.out.println(i);
    }
  }
}
//Verwendung:
Thread t = new Thread(new MyRunnable());
t.start();
\end{lstlisting}
\end{frame}

\begin{frame}[fragile]{Runnable}{Als Lambda Expression}
\lstset{style=java}
\begin{lstlisting}
Thread t = new Thread( () -> {
    for(int i=0;i<100;i++){
      System.out.println(i);
    }
  });
t.start()
\end{lstlisting}
\end{frame}

\begin{frame}{Runnable}{Aufgabe}
    \begin{alertblock}{}
    Implementiert zwei \texttt{Runnable} Klassen, die (verschiedene) Daten auf der Konsole ausgeben. Erzeugt dann in eurer \texttt{main()} Methode jeweils einen Thread zum Ausführen der Runnables und startet diese direkt nacheinander. 
    
    Was für eine Ausgabe erwartet ihr auf der Konsole?
    
    Inwiefern unterscheidet sich die tatsächliche Ausgabe von Eurer Erwartung?
    
    Beobachtet das Verhalten der Ausgaben bei mehrfacher Ausführung des Programms.
    
    \textit{Hinweis:} Für eine bessere Beobachtung ist es empfehlenswert, die Daten mehrfach über eine Schleife (>100 Iterationen) ausgeben zu lassen
    \end{alertblock}
\end{frame}
    
    \outlineSubframe{Eigenschaften und Zustände}
    \begin{frame}{Eigenschaften}{Von Threads}
    \begin{itemize}
        \item Beispiele für Eigenschaften von Threads:
        \begin{itemize}
            \item Priorität
            \item Zustand
            \item Name
        \end{itemize}
        \item Name lässt sich festlegen im Konstruktor:
        \begin{itemize}
            \item \texttt{Thread(Runnable r, String name)}
            \item \texttt{Thread(String name)}
        \end{itemize}
        \item Oder über die entsprechende Funktion:
        \begin{itemize}
            \item \texttt{void setName(String name)}
        \end{itemize}
    \end{itemize}
\end{frame}

\begin{frame}{Eigenschaften}{Zugriff auf die Eigenschaften}
    \begin{itemize}
        \item Analog gibt es auch eine \texttt{getName()} Funktion
        \item \textit{Erinnerung:} Die Methoden gehören zu der \texttt{Thread} Klasse!
        \item Nicht direkt in \texttt{Runnable} abrufbar!
        \item Jedoch lässt sich der Thread in dem gerade gearbeitet wird zurückgeben:
        \begin{itemize}
            \item \texttt{static Thread currentThread()}
            \item Aufruf: \texttt{Thread t = Thread.currentThread()}
        \end{itemize}
        \item Wenn der aktuelle Thread mehrfach benutzt wird (z.B. in einer Schleife) sollte dieser vor der Schleife abgerufen und zwischengespeichert werden
        \begin{itemize}
            \item \texttt{currentThread()} Aufruf ist "`teuer"'
        \end{itemize}
    \end{itemize}
\end{frame}

\begin{frame}{Zustände}{Von Threads}
    \begin{itemize}
        \item Lebenszyklus des Threads besteht aus verschiedenen Phasen:
        \begin{itemize}
            \item \textit{Nicht erzeugt}: Der Thread wurde schon definiert (über \texttt{new Thread(...)}), aber noch nicht gestartet
            \item \textit{Laufend}: Der Thread wurde gestartet (über die \texttt{start()} Methode)und wird aktuell durch den Prozessor bearbeitet
            \item \textit{Nicht Laufend}: Der Thread wurde gestartet (über die \texttt{start()} Methode), wird aber aktuell nicht bearbeitet
            \item \textit{Wartend}: Der Thread wartet auf ein bestimmtes Ereignis (Abschluss einer anderen Operation, Freiwerden einer Ressource...)
            \item \textit{Beendet}: Der Thread wurde fertig bearbeitet
        \end{itemize}
        \item Java bildet das über die \texttt{Thread.State} Enumeration ab
        \item Abrufbar über die \texttt{getState()} Methode
        \item Methode \texttt{isAlive()} gibt zurück, ob der Thread noch arbeitet oder beendet ist
    \end{itemize}
\end{frame}

\begin{frame}{Zustände}{In \texttt{Thread.State}}
    \begin{tabular}{|c|p{9cm}|}
    \hline
    \textbf{Zustand}&\textbf{Erklärung}\\
    \hline
    \hline
    \texttt{NEW} & Thread ist erzeugt, aber noch nicht gestartet\\\hline
    \texttt{RUNNABLE} & Thread wurde gestartet und läuft in der JVM\\\hline
    \texttt{BLOCKED} & Thread wartet auf das freiwerden eines bestimmten Locks um beispielsweise einen synchronisierten Codeabschnitt zu betreten\\\hline
    \texttt{WAITING} & Wartet etwa auf ein \texttt{notify()}\\\hline
    \texttt{TIMED\_WAITING} & Zeitgesteuertes Warten beispielsweise durch \texttt{sleep()}\\\hline
    \texttt{TERMINATED} & Thread wurde beendet\\\hline
    \end{tabular}
\end{frame}
    
    \outlineSubframe{Unterbrechen und Beenden}
    \begin{frame}{Unterbrechungen}{In Threads (Vgl. \cite{ullenboom2014java} S. 184f)}
    \begin{itemize}
        \item Threads bieten diverse Möglichkeiten zur Unterbrechung
        \item Simpelste: Die \texttt{sleep()} Methode
        \begin{itemize}
            \item Dieser wird (als \texttt{int}) die Zeit in Millisekunden übergeben, die pausiert werden soll
            \item Ist \textbf{nur} statisch vorhanden $\rightarrow$ Nur der eigene Thread lässt sich pausieren
            \item Kann eine \texttt{InterruptedException} auslösen, die über try-catch abgefangen werden muss
            \item Gleiche Funktionalität kann auch mit der \texttt{TimeUnit} Klasse erreicht werden $\rightarrow$ Dort eingängliche Definition der Zeit die gewartet wird
        \end{itemize}
    \end{itemize}
\end{frame}

\begin{frame}[fragile]{\texttt{sleep()}}{Codebeispiel}
\lstset{style=java}
\begin{lstlisting}
try {
  Thread.sleep(5000);
} catch(InterruptedException e) {}
//Alternativ:
try {
  TimeUnit.SECOND.sleep(5);
} catch(InterruptedException e) {}
\end{lstlisting}
\end{frame}

\begin{frame}{Verzicht}{Von Rechenzeit (Vgl. \cite{ullenboom2014java} S. 186f)} 
    \begin{itemize}
        \item Ein Thread kann von sich aus auf Rechenzeit verzichten
        \item Durch Aufruf von \texttt{Thread.yield()};
        \item Signalisiert dem Betriebssystem, dass auf weitere ggf. bereitgestellte Zeit verzichtet wird und direkt auf einen anderen Prozess geswitched werden kann
        \item \textbf{Nicht deterministisch} -- Wann der Thread wieder aufgenommen wird ist unklar
        \item Daher mit Bedacht zu verwenden
        \item Oft nur für sehr spezielle Anwendungsfälle wirklich praktikabel
    \end{itemize}
\end{frame}

\begin{frame}{Unterbrechen}{Über Interrupt (Vgl. \cite{ullenboom2014java} S. 189ff)}
    \begin{itemize}
        \item Häufig laufen in nebenläufigen Threads Anwendungen in Endlosschleifen um dauerhaft verfügbar zu sein
        \begin{itemize}
            \item Beispielsweise Server, die auf ankommende Verbindungen warten
            \item Durch die Endlosschleife würden diese auf natürliche Weise nicht beenden
        \end{itemize}
        \item Deshalb kann von außen an den Thread eine \textit{Anfrage auf Unterbrechung} gestellt werden
        \item Über die Methode \texttt{interrupt()}
        \item Anders als der Name ggf. vermuten lässt wird der Thread nicht direkt unterbrochen
        \item Es wird lediglich ein Flag im Thread gesetzt, die dieser eigenständig überprüfen muss
        \item Thread kann selbst über \texttt{isInterrupted()} prüfen, ob er unterbrochen wurde
    \end{itemize}
\end{frame}

\begin{frame}[fragile]{Interrupt}{Codebeispiel}
\lstset{style=java}
\begin{lstlisting}
Thread t = new Thread(()->{
  while(!isInterrupted()){
    System.out.println("Running...");
    try{
      Thread.sleep(500);
    } catch (InterruptedException e){
      interrupt();
      System.out.println("Interrupt caught while sleep()");
    }
  }
  System.out.println("Ende");
});
t.start();
Thread.sleep(2000);
t.interrupt();
\end{lstlisting}
\end{frame}

\begin{frame}{Daemons}{Speziell für Endlosthreads (Vgl. \cite{ullenboom2014java} S. 187f)}
    \begin{itemize}
        \item Für "`Endlosthreads"' im Hintergrund gibt es einen speziellen Modi
        \item Man kann den Thread als \textit{Daemon} markieren
        \item Das bedeutet: Der Thread wird solange ausgeführt, solange noch andere Threads (die keine Daemons sind) laufen
        \item Sollten nur noch Daemon Threads laufen, so werden diese alle beendet
        \item Markierung als Daemon über die Methode \texttt{setDaemon(boolean)}
    \end{itemize}
\end{frame}

\begin{frame}{Rendezvous}{Zusammenführen von Threads (Vgl. \cite{ullenboom2014java} S. 194f)}
    \begin{itemize}
        \item Threads werden unter anderem genutzt, um nebenläufige Berechnungen durchzuführen
        \item Die Ergebnisse werden an späterer Stelle benötigt
        \item Zum Verwendungszeitpunkt muss sichergestellt sein, dass die Berechnung abgeschlossen ist
        \item Dafür wird die Methode \texttt{join()} genutzt
        \item Diese blockiert die Ausführung so lange, bis der entsprechende Thread beendet ist (Die \texttt{run()} Methode beendet wurde)
    \end{itemize}
\end{frame}

\begin{frame}[fragile]{\texttt{join()}}{Codebeispiel}
\lstset{style=java}
\begin{lstlisting}
public class JoinThread extends Thread{
  public int result =0;
  @Override
  public void run(){
    result=1;
  }
}
//Anwendung
JoinThread t = new JoinThread();
t.start();
//t.join();
System.out.println(t.result);
\end{lstlisting}
\end{frame}

\begin{frame}{Rendezvous}{Überladungen}
    \begin{itemize}
        \item Über verschiedene Überladungen lässt sich eine maximale Wartezeit definieren:
        \begin{itemize}
            \item \texttt{join(long millis)} -- Definiert eine Wartezeit in Millisekunden
            \item \texttt{join(long millis, long nanos)} -- Definiert eine Wartezeit in Milli- und Nanosekunden
        \end{itemize}
        \item Nach \texttt{join} Aufruf mit Timeout kann über \texttt{isAlive()} geprüft werden, ob die Berechnung wirklich abgeschlossen wurde
    \end{itemize}
\end{frame}
    
    \outlineFrame{Executors}
    \outlineSubframe{Callable}
    \outlineSubframe{Future}
    
    
    \printbibliographyframe
    
	\section*{Kontakt}
	\begin{frame}{Kontakt}{}
	\begin{itemize}
		\item E-Mail: \href{mailto:lukas.abelt@airbus.com}{lukas.abelt@airbus.com}
		\item GitHub: \url{https://www.github.com/LuAbelt}
		\item GitLab: \url{https://www.gitlab.com/LuAbelt}
		\item Telefon(Firma): 07545 - 8 8895
		\item Telegram: LuAbelt
	\end{itemize}
\end{frame}
	

\end{document}