%%%%%%%%%%%%%%%%%%%%%%%%%%%%%%%%%%%%%%%%%%%%%%%%%%%%%%%%%%%%%%%%%%%%%%%%%%%%%%%%%%%%%%%%%%
%%
%% Description:		This is an example presentation using the beamerthemedhbw
%%
%%					The beamerthemedhbw is based on jacksbeamertheme
%%					(https://github.com/JacknJo/jacksbeamertheme)
%%
%% Author:			Hannes Bartle																				
%% 					DHBW Ravensburg Campus Friedrichshafen		
%%					September 2016	
%% 
%% The beamerthemedhbw is free software: you can redistribute it and/or modify
%% it under the terms of the GNU General Public License as published by
%% the Free Software Foundation, either version 3 of the License, or
%% (at your option) any later version.
%% 
%% The beamerthemedhbw is distributed in the hope that it will be useful,
%% but WITHOUT ANY WARRANTY; without even the implied warranty of
%% MERCHANTABILITY or FITNESS FOR A PARTICULAR PURPOSE.  See the
%% GNU General Public License for more details.
%% 
%% You should have received a copy of the GNU General Public License
%% along with the beamerthemedhbw.  If not, see <http://www.gnu.org/licenses/>.
%% 
%% 
%%%%%%%%%%%%%%%%%%%%%%%%%%%%%%%%%%%%%%%%%%%%%%%%%%%%%%%%%%%%%%%%%%%%%%%%%%%%%%%%%%%%%%%%%%


\documentclass[	12pt, 				
				t,					
				aspectratio=169,
				%handout-PLACEHOLDER
				]{beamer}

\usepackage{dhbwstyle}

\title{Java GUI Grundlagen}

\begin{document}
	
	\begin{frame}[noframenumbering]
		\titlepage
	\end{frame}


	\begin{frame}{Inhalt}
		\tableofcontents
	\end{frame}
    
    \outlineFrame{Historie}
    \begin{frame}{Java GUI Anwendungen}{Die Historie}
    \begin{itemize}
        \item
        \item
        \item
        \item
        \item
    \end{itemize}
\end{frame}
    
    \outlineFrame{Grundlegender Aufbau}
    \begin{frame}{Swing}{Grundlegendes}
	\begin{itemize}
		\item AWT Komponenten erzeugen Komponenten über Betriebssystemaufrufe
		\item Dadurch Speicher nicht über Java verwaltet
		\item Swing erzeugt Komponenten über direkte Low-Level Calls
		\item Jede Komponente wird auf den Bildschirm gezeichnet
		\item Dadurch komplette Speicherverwaltung in Java
		\item Overhead führt ggf. zu Performanceverlust
		\item Aber mehr Kontrolle über die Features
	\end{itemize}
\end{frame}

\begin{frame}{Swing}{Konzepte}
	\begin{itemize}
		\item Swing nutzt das Entwurfstmuster des \textit{Kompositums}
		\item Kompositum beschreibt im Grunde eine Art Baumstruktur
	\end{itemize}
	%TODO: Abbildungen Kompositum&Klassendiagramm vom Judt
\end{frame}
    
    \outlineFrame{Grundlegende Komponenten}
    \begin{frame}{Grundlegende Komponenten}
\end{frame}
	
    \outlineFrame{Layout Manager}
    \begin{frame}{Links, Rechts, Oben, Links!}{Verwenden von Layout Managern}
\end{frame}
    
    \outlineFrame{Eigene Komponenten}
    \begin{frame}{Eigene Fenster}{Erstellen von Fenstern}
    \begin{itemize}
        \item Unsere bisherigen Fenster haben wir uns in der \texttt{main} zusammengebastelt
        \item Dies ist in der Regel unpraktikabel
        \item Code ist nicht wiederverwendbar
        \item Und ggf. schwierig wartbar
        \item Angestrebt wird eine Trennung von Darstellung und Logik und Daten
        \item Daher: Kapselung von GUI in eigene Klasse
        \begin{itemize}
            \item Diese erbt von \texttt{JFrame}
            \item Fenster wird im Konstruktor "`zusammengebaut"'
            \item Hinzufügen aller Komponenten, Layouten von diesen, setzen von Attributen wie Größe etc.
            \item In der Main Klasse wird dann lediglich ein Objekt von unserer neuen \texttt{JFrame} Unterklasse erzeugt
        \end{itemize}
    \end{itemize}
\end{frame}

\begin{frame}{Eigene Komponenten}{Grundlegendes}
    \begin{itemize}
        \item Erben vom \texttt{JComponent} Typ
        \item Lassen sich vielseitig konfigurieren z.B. in:
        \begin{itemize}
            \item Aussehen
            \item Verhalten
        \end{itemize}
        \item Nützlich kann es zum Beispiel sein die \texttt{getXxxSize} Methode zu überschreiben, wenn eine Komponente immer eine bestimmte Größe haben soll
        \item Fokus heute: Zeichnen von eigenen Komponenten
    \end{itemize}
\end{frame}

\begin{frame}{Zeichnen von Komponenten}{Schritte im \texttt{paint()} Prozess}
    \begin{itemize}
        \item Jede Komponente besitzt eine \texttt{paint()} Methode
        \item In dieser wird das Aussehen der Komponente auf den Bildschirm gezeichnet
        \item \texttt{paint()} ruft drei Methoden (in dieser Reihenfolge) auf:
        \begin{itemize}
            \item \texttt{paintComponent(Graphics g)}
            \item \texttt{paintBorder(Graphics g)}
            \item \texttt{paintChildren(Graphics g)}
        \end{itemize}
        \item In der Regel reicht es aus \texttt{paintComponent()} zu überschreiben
        \item Wir rufen \texttt{paint()} bzw. \texttt{paintComponent()} \textbf{nie} direkt auf
    \end{itemize}
\end{frame}

\begin{frame}{Zeichnen von Komponenten}{Die \texttt{paint()} Methode}
    \begin{itemize}
        \item \texttt{paint()} Wird durch das System automatisch aufgerufen bei zB.:
        \begin{itemize}
            \item Verschieben des Fensters/Von Komponenten
            \item Ändern der Fenstergröße
            \item Minimieren/Maximieren
        \end{itemize}
        \item Wenn programmatisch neu gezeichnet werden muss (Zum beispiel weil sich Daten geändert haben) wird die \texttt{repaint()} Methode aufgerufen
        \item \texttt{repaint()} akzeptiert Argumente, sodass nur ein bestimmter Bereich neu gezeichnet wird
    \end{itemize}
\end{frame}

\begin{frame}{Das \texttt{Graphics} Objekt}{Zeichnen von Komponenten}
    \begin{itemize}
        \item Den \texttt{paintXXX()} Methoden wird ein Objekt vom Typ \textit{Graphic} übergeben
        \item Dieses fasst verschiedene Aspekte zum Zeichnen zusammen:
        \begin{itemize}
            \item Aktuelle Komponente auf die gezeichnet wird
            \item Die Aktuelle Farbe
            \item Die aktuelle Font
            \item Weitere Aspekte die für die aktuelle Betrachtung nicht von Relevanz sind
        \end{itemize}
        \item Die Graphics Klasse bietet einige Funktionen zum Zeichnen von primitiven Formen an
    \end{itemize}
\end{frame}

\begin{frame}{Das \texttt{Graphics} Objekt}{Zeichenoperationen}
    \begin{itemize}
        \item \texttt{drawString} -- Zeichnet einen gegebenen String an der angegebenen Position (Mehrere Überladungen)
        \item \texttt{drawImage} -- Zeichnet ein Bild an einer bestimmten Stelle (Mehrere Überladungen)
        \item \texttt{drawArc} -- Zeichnet einen Ellipsenbogen
        \item \texttt{drawLine} -- Zeichnet eine Linie zwischen zwei Punkten
        \item \texttt{drawOval} -- Zeichnet ein Oval
        \item \texttt{drawRect} -- Zeichnet ein Rechteck
        \item \texttt{drawRoundRect} -- Zeichnet ein abgerundetes Rechteck
    \end{itemize}
\end{frame}

\begin{frame}{Zeichnen von Komponenten}{Weitere nützliche Operationen}
    \begin{itemize}
        \item Über \texttt{getWidth()} bzw. \texttt{getHeight()} kann die aktuelle Größe der Komponenten bestimmt werden
        \item Für viele der \texttt{drawXXX} Methoden existieren analoge \texttt{fillXXX} Methoden
        \item Aktuelle Zeichenfarbe lässt sich ändern über \texttt{setColor}
        \item Analog lässt sich \texttt{setFont} verwenden um die Schriftart zu ändern
    \end{itemize}
\end{frame}

\begin{frame}{Hinweise}{Beim Entwerfen von View-Komponenten}
    \begin{itemize}
        \item Die Komponenten sollten nur für die Darstellung verantwortlich sein
        \item In ihnen werden in der Regel keine Applikationsdaten gespeichert
        \item Auch die Logik ist in der Regel entkoppelt von der Darstellung
        \item Man spricht hierbei auch vom \textit{Model-View-Controller (MVC)} Konzept:
        \begin{itemize}
            \item Model -- Tatsächlich verwendetes Datenmodell. Speichert die Applikationsdaten (Datenschicht)
            \item View -- Anzeigen/rendern von Daten. Dient lediglich zur Darstellung (Präsentationsschicht)
            \item Controller -- Realisiert die Steuerung in Interaktion des Nutzers. Bildet in der Regel die Kopplung zwischen Model und View (Interaktionsschicht)
        \end{itemize}
    \end{itemize}
\end{frame}
    
    \outlineFrame{Aufgaben}
    \begin{frame}{Aufgabe 1}{Größter Teiler}
Entwerft einen Algorithmus, der für einen gegebene natürliche Zahl den größten (ganzzahligen) Teiler findet. Erstellt zu eurem Algorithmus einen Programmablaufplan oder Struktogramm.

Analysiert euren Algorithmus bezüglich der Komplexität $O$. Vergleicht euren Algorithmus mit denen der anderen.
\end{frame}

\begin{frame}[fragile]{Aufgabe 2}{Codeanalyse}
Analysiert den folgenden Codeausschnitt. Welche Aufgabe hat der Algorithmus? Bestimmt die Komplexität $O$.

\lstset{style=java}
\begin{lstlisting}
public void func(List<?> in1, List<?> in2){
  for(int i=0;i<in1.size();i++){
      for(int j=0;in2.size();j++){
          String out = String.format("{\%s,\%s}", in1.get(i), in2.get(j));
          System.out.println(out);
      }
  }
}
\end{lstlisting}
\end{frame}

\begin{frame}{Aufgabe 3}{}
%TODO: Primzahlenalgorithmus als PAP. Erkennen, was er macht und umsetzen in Code
\end{frame}
    
    \printbibliographyframe
    
	\section*{Kontakt}
	\begin{frame}{Kontakt}{}
	\vfill
	\begin{itemize}
		\item E-Mail: \texttt{lukas.abelt@airbus.com}
		\item GitHub: \url{https://www.github.com/LuAbelt}
		\item GitLab: \url{https://www.gitlab.com/LuAbelt}
		\item Telefon(Firma): 07545 - 8 8895
		\item Telegram: LuAbelt
	\end{itemize}
	\vfill
\end{frame}
	

\end{document}