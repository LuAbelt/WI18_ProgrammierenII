\begin{frame}{Aufgabe 1}{Simple Komponente}
\begin{alertblock}{}
Entwickelt eine simple Komponente als Unterklasse von \texttt{JComponent}. Überschreibt die \texttt{paintComponent} Methode und experimentiert mit den verschiedenen Methoden des \texttt{Grpahics} Objektes
\end{alertblock}
\end{frame}

\begin{frame}{Aufgabe 2}{Sinuskurve}
\begin{alertblock}{}
Entwickelt eine Komponente, die eine Sinuskurve zeichnet. Die Sinuskurve sollte bei vergrößern/verkleinern des Fensters entsprechend weiter gezeichnet werden. Die x-Achse sollte in der Mitte der Komponenten liegen und die maximale Amplitude noch im Bereich
der Komponenten liegen.

\textbf{Hinweis:} Um den Sinus einer bestimmten Zahl zu bestimmen könnt ihr die Math.sin(double) Funktion verwenden. Beachtet allerdings, dass diese einen Wert im Intervall $[-1;1]$ zurück gibt, den ihr ggf. für die Größe skalieren müsst
\end{alertblock}
\end{frame}