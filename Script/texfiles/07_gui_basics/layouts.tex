\begin{frame}{GUI Layout}{Anordnen von Elementen}
    \begin{itemize}
        \item Bei Nutzen von \texttt{add()} werden Komponenten automatisch platziert
        \item Dies geschieht durch den \textit{Layout Manager}
        \item Manager werden pro Container definiert
        \item Je nach Layout sind ggf. feste "`Slots"' im Container vorhanden
        \item Um Komponente an bestimmter Position zu platzieren wird Überladung von \texttt{add()} verwendet:
        \begin{itemize}
            \item \texttt{add(Component comp, Object constraint)}
            \item \texttt{constraint} definiert Position im Container
        \end{itemize}
    \end{itemize}
\end{frame}

\begin{frame}{GUI Layout}{Größe von Komponenten}
    \begin{itemize}
        \item Komponenten können absolute, minimale, maximale und bevorzugte Größe definieren
        \begin{itemize}
            \item \texttt{setSize(int height, int width)}
            \item \texttt{setMinimumSize(Dimension d)}
            \item \texttt{setMaximumSize(Dimension d)}
            \item \texttt{setPreferredSize(Dimension d)}
        \end{itemize}
        \item Layout Manager \textit{versucht} diese zu beachten
        \item \textit{Müssen} dies jedoch nicht
        \item Insbesondere absolute Größen sind meist gegen das Prinzip von Layout Managern
    \end{itemize}
\end{frame}

\begin{frame}{GUI Layout}{Layout Manager}
    \begin{itemize}
        \item Vorteil von Layout Managern:
        \begin{itemize}
            \item Relative Positionierung von Komponenten kann definiert werden
            \item Vereinfachen erstellen von GUIs
            \item Größe und Position von Komponenten werden automatisch angepasst (Zum Beispiel bei Ändern der Fenstergröße)
        \end{itemize}
    \end{itemize}
\end{frame}

\begin{frame}{BorderLayout}{Grundlegendes (Vgl. \cite{orac:borderlayout})}
    \begin{itemize}
        \item Teil den Container in fünf Bereiche
        \item Diese sind über Konstanten in der \texttt{BorderLayout} Klasse definiert:
        \begin{itemize}
            \item \texttt{PAGE_START} -- Oberer Bereich (Kopfzeile)
            \item \texttt{PAGE_END} -- Unterer Bereich (Fußzeile)
            \item \texttt{LINE_START} -- Linker Bereich zwischen Kopf und Fuß
            \item \texttt{LINE_END} -- Rechter Bereich zwischen Kopf und Fuß
            \item \texttt{CENTER} -- Mittlerer (Haupt-)bereich
        \end{itemize}
        \item Beispiel: Hinzufügen im \texttt{PAGE_START} Bereich:
        \begin{itemize}
            \item \texttt{add(button, BorderLayout.PAGE_START)}
        \end{itemize}        
    \end{itemize}
\end{frame}

\begin{frame}{BorderLayout}{Weitere Eigenschaften (Vgl. \cite{orac:borderlayout})}
    \begin{itemize}
        \item Fokus liegt auf den \texttt{CENTER} Bereich
        \item Dieser nutzt maximal verfügbaren Platz
        \item Alle anderen Bereiche nutzen nur den benötigten Platz
        \item Nicht alle Bereiche müssen genutzt werden
        \item \texttt{PAGE_START} und \texttt{PAGE_END} nutzen (sofern vorhanden) immer die volle Breite
        \item Freiräume zwischen Bereichen lassen sich definieren:
        \begin{itemize}
            \item \texttt{setHGap(int)}
            \item \texttt{setVGap(int)}
        \end{itemize}
        \item Standard-Layoutmanager von \texttt{JFrame}
    \end{itemize}
\end{frame}

\begin{frame}{BorderLayout}{Aufgabe}
    \begin{alertblock}{BorderLayout Demo}
    Implementiert eine simple Fensteranwendung, um das Verhalten des \texttt{BorderLayout} kennen zu lernen. Platziert dazu zunächst jeweils einen \texttt{JButton} in
    jedem Bereich. 
    
    Untersucht, wie sich die Größe und Anordnung der Buttons ändert, wenn ihr eine Größe für die Buttons definiert oder wenn ihr einzelne Bereiche im Layout ungenutzt lasst.
    \end{alertblock}
\end{frame}

\begin{frame}{BoxLayout}{Grundlegendes(Vgl. \cite{orac:boxlayout})}
    \begin{itemize}
        \item Ordnet Komponenten entweder vertikal oder horizontal angeordnet
        \item Dafür zwei Konstanten in der \texttt{BoxLayout} Klasse
        \begin{itemize}
            \item \texttt{PAGE_AXIS} -- Vertikale Anordnung (Als eine "`Spalte"')
            \item \texttt{LINE_AXIS} -- Horizontale Anordnung (Als eine "`Zeile"')
        \end{itemize}
        \item Komponenten werden dann gemäß \texttt{ComponentAlignment} platziert (Bspw. Linksbündig/Zentriert/Rechtsbündig)
        \item Definierte maximale Größe von Komponenten wird im \texttt{BoxLayout} \texit{immer} respektiert
    \end{itemize}
\end{frame}

\begin{frame}[fragile]{BoxLayout}{Aufgabe}
Entwerft eine Beispielandwendung, in der mehrere Buttons in einem Frame mit dem \texttt{BoxLayout} als Layout Manager. Nutzt folgende Basis um den Layout Manager im \texttt{JFrame()} anzupassen:

\lstset{style=java}
\begin{lstlisting}
JFrame frame = new JFrame();
frame.setSize(300,600);
frame.setLayout(new BoxLayout(window, BoxLayout.PAGE_AXIS));

/*Buttons hinzufügen*/

frame.setVisible(true);
\end{lstlisting}
\end{frame}

\begin{frame}{BorderLayout}{}
\end{frame}

\begin{frame}{GridLayout}{}
\end{frame}

\begin{frame}{GridBagLayout}{}
\end{frame}

