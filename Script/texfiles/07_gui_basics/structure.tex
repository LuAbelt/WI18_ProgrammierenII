\begin{frame}{Swing}{Grundlegendes}
	\begin{itemize}
		\item AWT Komponenten erzeugen Komponenten über Betriebssystemaufrufe
		\item Dadurch Speicher nicht über Java verwaltet
		\item Swing erzeugt Komponenten über direkte Low-Level Calls
		\item Jede Komponente wird auf den Bildschirm gezeichnet
		\item Dadurch komplette Speicherverwaltung in Java
		\item Overhead führt ggf. zu Performanceverlust
		\item Aber mehr Kontrolle über die Features
	\end{itemize}
\end{frame}

\begin{frame}{Swing}{Konzepte}
	\begin{itemize}
		\item Swing nutzt das Entwurfstmuster des \textit{Kompositums}
		\item Kompositum beschreibt im Grunde eine Art Baumstruktur
		\item Objekte werden zu Gruppen zusammengefasst und Gruppen wiederum zu größeren Gruppen
		\item Dadurch kann eine einheitliche Behandlung von Objekten und Aggregaten erreicht werden
		\item Gemeinsame Eigenschaften von Objekt und Aggregat werden in einer Oberklasse isoliert
	\end{itemize}
\end{frame}

\begin{frame}{Beispiel für Kompositum}{Zusammengesetzte Grapfik-Objekte (Vgl. \cite{judt2017} S. 26}
    \begin{figure}
	\includegraphics*[width=.8\textwidth]{graph/compositum_example}
    \end{figure}
	Gemeinsame Operationen: \texttt{zeichne()}, \texttt{verschiebe()}, \texttt{lösche()}, \texttt{skaliere()}
\end{frame}

\begin{frame}{Kompositum}{Beispielhaftes Klassendiagramm}
    \begin{figure}
	\includegraphics*[width=.8\textwidth]{graph/compositum_cd}
    \caption*{Quelle: \cite{judt2017}}
    \end{figure}
\end{frame}

\begin{frame}{Kompositum}{Umsetzung in Swing}
	\begin{itemize}
		\item Die Klassenhierarchie stellt ein Kompositum dar
		\item Swing Objekte unterteilen sich hierbei in zwei Klassen:
		\begin{itemize}
			\item \texttt{Component}
			\item \texttt{Container}
		\end{itemize}
		\item \texttt{Container} fassen \texttt{Components} zusammen
		\item Jeder Container ist für die Anordnung seiner Komponenten zuständig
		\item Jeder \texttt{Container} ist auch immer eine \texttt{Component}
	\end{itemize}
\end{frame}

\begin{frame}{Klassendiagramm}{Der Swing Bibliothek}
    \begin{figure}
	\includegraphics*[width=.8\textwidth]{graph/swing_cd}
    \end{figure}
\end{frame}

\begin{frame}{Swing Fenster}{JFrame Klasse}
	\begin{itemize}
		\item Basis eines Swing Fensters ist die \texttt{JFrame} Klasse
		\item Basiert auf der AWT Klasse \texttt{Frame}
		\item Lässt sich manipulieren z.B. in bezug auf
		\begin{itemize}
			\item Sichtbarkeit
			\item Größe und Position
			\item Operation beim schließen
		\end{itemize}
		\item Leeres Fenster wird erzeugt über den \texttt{JFrame()} Konstruktor
		%TODO: irgendwas irgendwas JFrame als basis
		%Referenz JFrame zu Frame?
	\end{itemize}
\end{frame}

\begin{frame}[fragile]{Swing Fenster}{Ganz simpel}
\lstset{style=java}
\begin{lstlisting}
public static void main(String[] args){
  JFrame window = new JFrame();
  window.setDefaultCloseOperation(WindowConstants.EXIT_ON_CLOSE);
  window.setVisible(true);
}
\end{lstlisting}
\end{frame}

\begin{frame}{JFrame}{Ändern der Größe des Fensters}
    \begin{itemize}
        \item Größe wird über \texttt{setSize()} verändert
        \item Hier gibt es zwei Überladungen:
        \begin{itemize}
            \item Als Parameter ein \texttt{Dimension} Objekt
            \item Zwei \texttt{int} Werte als Parameter
        \end{itemize}
    \end{itemize}
\end{frame}

\begin{frame}{Swing Fenster}{Weitere Eigenschaften}
    \begin{itemize}
        \item Bildet einen \textit{Top-Level-Container}
        \item \texttt{JFrame} wird immer mit Titel und Menüleiste erzeugt
        \item Erscheinung des Frames kann geändert werden
        \begin{itemize}
            \item Man spricht in der Regel von "`Decorations"'
            \item Ändern der Form des Fensters (In begrenztem Rahmen)
            \item Wählen verschiedener Farbschema
            \item (Teil-)transparente Fenster
            \item usw
        \end{itemize}
    \end{itemize}
\end{frame}

\begin{frame}{Weitere Fenster in Swing}
    \begin{itemize}
        \item \texttt{JWindow} -- Fenster ohne Menüleiste
        \item \texttt{JDialog} -- Zum erstellen von (modalen) Dialogfeldern
    \end{itemize}
\end{frame}