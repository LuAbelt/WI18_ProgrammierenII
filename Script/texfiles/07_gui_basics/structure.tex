\begin{frame}{Swing}{Grundlegendes}
	\begin{itemize}
		\item AWT Komponenten erzeugen Komponenten über Betriebssystemaufrufe
		\item Dadurch Speicher nicht über Java verwaltet
		\item Swing erzeugt Komponenten über direkte Low-Level Calls
		\item Jede Komponente wird auf den Bildschirm gezeichnet
		\item Dadurch komplette Speicherverwaltung in Java
		\item Overhead führt ggf. zu Performanceverlust
		\item Aber mehr Kontrolle über die Features
	\end{itemize}
\end{frame}

\begin{frame}{Swing}{Konzepte}
	\begin{itemize}
		\item Swing nutzt das Entwurfstmuster des \textit{Kompositums}
		\item Kompositum beschreibt im Grunde eine Art Baumstruktur
		\item Objekte werden zu Gruppen zusammengefasst und Gruppen wiederum zu größeren Gruppen
		\item Dadurch kann eine einheitliche Behandlung von Objekten und Aggregaten erreicht werden
		\item Gemeinsame Eigenschaften von Objekt und Aggregat werden in einer Oberklasse isoliert
	\end{itemize}
	%TODO: Abbildungen Kompositum&Klassendiagramm vom Judt
\end{frame}

\begin{frame}{Beispiel für Kompositum}{Zusammengesetzte Grapfik-Objekte (Vgl. \cite{judt2017} S. 26}
	\includegraphics*[width=.8\textwidth]{graph/compositum_example}
	Gemeinsame Operationen: \texttt{zeichne()}, \texttt{verschiebe()}, \texttt{lösche()}, \texttt{skaliere()}
\end{frame}

\begin{frame}{Kompositum}{Beispielhaftes Klassendiagramm}
	\includegraphics*[width=.8\textwidth]{graph/compositum_cd}
\end{frame}

\begin{frame}{Kompositum}{Umsetzung in Swing}
	\begin{itemize}
		\item Die Klassenhierarchie stellt ein Kompositum dar
		\item Swing Objekte unterteilen sich hierbei in zwei Klassen:
		\begin{itemize}
			\item \texttt{Component}
			\item \texttt{Container}
		\end{itemize}
		\item \texttt{Container} fassen \texttt{Components} zusammen
		\item Jeder Container ist für die Anordnung seiner Komponenten zuständig
		\item Jeder \texttt{Container} ist auch immer eine \texttt{Component}
	\end{itemize}
\end{frame}

\begin{frame}{Klassendiagramm}{Der Swing Bibliothek}
	\includegraphics*[width=.8\textwidth]{graph/swing_cd}
\end{frame}

\begin{frame}{Swing Fenster}
	\begin{itemize}
		\item 
		%TODO: irgendwas irgendwas JFrame als basis
		%Referenz JFrame zu Frame?
		%Panel als Grundelement
	\end{itemize}
\end{frame}