\begin{frame}{Komponenten}{Grundlegendes}
    \begin{itemize}
        \item Komponenten in Swing sind frei erweiterbar
        \item Swing Komponenten i.D.R. über Namensgebung von AWT Komponenten unterscheidbar
        \item Für viele Anwendungsfälle sind jedoch die Standardkomponenten schon ausreichen
        \item Neue Komponenten werden über \item{add()} Methode zu einem beliebigen Container hinzugefügt
        \item Komponenten können eine bevorzugte oder minimale Größe definieren
        \item Position wird i.d.R. automatisch bestimmt (Über den Layoutmanager)
    \end{itemize}
\end{frame}

\begin{frame}[allowframebreaks]{Grundlegende Komponenten}{Übersicht}
    \begin{itemize}[<+->]
        \item \texttt{JPanel} -- Einfacher Bereich ohne spezielle Besonderheiten. Ist ein \texttt{Container} und wird in der Regel genutzt um Elemente zu gruppieren und ggf. anzuordnen
        \item \texttt{JButton} -- Einfacher Button für den eine Funktion programmiert werden kann(Zum Beispiel das Starten einer Berechung)
        \item \texttt{JCheckBox} -- Auswahlbox für eine binäre Eingabe
        \item \texttt{JLabel} -- Simples Feld zum Anzeigen von Texten. Kann nicht durch den Nutzer bearbeitet werden
        \item \texttt{JTextField} -- Textfeld zum Anzeigen oder Bearbeiten von Texten (Einzelzeilen). Kann durch den Nutzer verändert werden
        \item \texttt{JSlider} -- Schieberegler
        \item \texttt{JRadioButton} -- Gibt die Möglichkeit zur Auswahl von einer Option aus mehreren gegebenen Möglichkeiten. Es kann pro Gruppe an Optionen immer nur genau eine gleichzeitig aktiviert sein.
        \item \texttt{JComboBox} -- Element, das die Auswahl eines Eintrags aus einer Liste erlaubt oder auch einen eigenen Eintrag ermöglicht
        \item \texttt{JEditorPane} -- Ermöglicht die Eingabe von mehrzeiligen Texten
    \end{itemize}
\end{frame}

