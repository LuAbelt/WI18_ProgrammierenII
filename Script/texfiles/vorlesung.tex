
	% Allgemeines
	%	Script x
	%	Organisation x
	
	% Ziele x
	%	Laut Modulbeschreibung x
	%	Mein Ziel x
	
	% Themen & Termine x
	
	% Prüfungsleistung

\begin{frame}{Allgemeines}{Skript}
	\begin{itemize}
		\item Mit \LaTeX{} erstellt
		\item Im Druck verfügbar...
		\begin{itemize}
			\item ...wenn man bezahlt hat
		\end{itemize}
		\item Digitale Version als PDF verfügbar
		\item Source Code zum selbst compilen auf GitHub verfügbar:
		\begin{itemize}
			\item \small{\url{https://github.com/LuAbelt/WI18\_ProgrammierenII}}
		\end{itemize}
	\end{itemize}
	\begin{block}{Git repository clonen}
		\tt{git clone https://github.com/LuAbelt/WI18\_ProgrammierenII}
	\end{block}
\end{frame}

\begin{frame}{Allgemeines}{Skript}
	{\small \setlength{\forkmeoffset}{2.5cm} \forkme[east]}
	\begin{minipage}{\textwidth}
		\centering	
		\includegraphics[height=0.6\textheight]{graph/gitqr.png}
	\end{minipage}
\end{frame}

\begin{frame}{Allgemeines}{Organisation}
	\begin{itemize}
		\item Insgesamt 60 UE über 15 Termine $\rightarrow$ 4 UE pro Termin
		\item Also 4 UE pro Termin
		\item Aufteilung (Richtlinie)
		\begin{itemize}
			\item 2 UE Theorie
			\item Kaffeepause
			\item 2 UE praktische Anwendung
		\end{itemize}
	\end{itemize}
\end{frame}

\begin{frame}{Allgemeines}{Organisation}
	\begin{itemize}
		\item Zu (fast) jedem Termin wird es eine praktische Aufgabe zur Implementierung geben
		\item Aufgaben \& meine Beispielimplementierung: Nach der Vorlesung im Git Repo zu finden
		\begin{itemize}
			\item Als IntelliJ \& Eclipse Projekte
			\item Warum IntelliJ? - Ich arbeite mit IntelliJ (Und würde es auch jedem empfehlen)
			\item Warum Eclipse? - DHBW Rechner haben nur Eclipse installiert
			\item IntelliJ kann Projekte in IntelliJ Projekte exportieren
		\end{itemize}
	\end{itemize}
\end{frame}

\begin{frame}{Allgemeines}{Feedback}
	\begin{itemize}
		\item Auch für mich neu
		\item Fragen gerne immer und sofort
		\item Feedback gerne über alle Kanäle wie zum Beispiel:
		\begin{itemize}
			\item Persönlich
			\item Per Mail
			\item Über Github Issues
			\item Per Telefon
			\item usw...
		\end{itemize}
		\item Kontakdaten am Ende jedes Foliensatzes
	\end{itemize}
\end{frame}

\begin{frame}[allowframebreaks]
	\frametitle{Ziele der Vorlesung}
	\framesubtitle{Laut Modulbeschreibung}
	
	\vfill
	\begin{block}{Fachkompetenz}
		Die Studierenden kennen fortgeschrittene Konzepte objektorintierter Programmiersprachen. Sie besitzen Kenntnisse über wichtige Algorithmen und Datenstrukturen sowie Methoden
		zur Beurteilung der Effizienz und Qualität von Algorithmen.
	\end{block}
	\vfill
	
	\framebreak
	
	\vfill
	\begin{block}{Methodenkompetenz}
		Die Studierenden können fortgeschrittene Konzepte der Objektorientierung anwenden und autonom mittlere bis größere lauffähige Programme implementieren und testen. Sie sind in der Lage, Algorithmen verschiedener Darstellungsarten zu 
		verstehen und ihre Effizienz bzw. Qualität zu beurteilen, aber auch selbstständig Algorithmen und dazu erforderliche Datenstrukturen zu entwickeln und zu implementieren.
	\end{block}
	\vfill
	
	\framebreak
	
	\vfill
	\begin{block}{Personale und Soziale Kompetenz}
		Die Studierenden können eigenständig Algorithmen und Lösungsverfahren erarbeiten. Sie können stichhaltig und sachangemessen über Konzepte und eigene Algorithmen und deren Implementierung 
		und die damit verbundenen Probleme argumentieren, eigene Umsetzungen plausibel darstellen und eventuelle Fehler nachvollziehbar gegenüber anderen begründen.
	\end{block}
	\vfill
	
	\framebreak
	
	\vfill
	\begin{block}{Übergreifende Handlungskompetenz}
		Die Studierenden können unter Einsatz der Programmiersprache komplexe praktische Probleme modellieren, algorithmisch behandeln und in anwenderfreundliche und effiziente Lösungen umsetzen. Sie 
		können praktische Problemstellungen analysieren und bekannte Algorithmen und Datenstrukturen effizienzorientiert darauf anwenden und falls notwendig an die Problemstellung anpassen.
	\end{block}
	\vfill
	
	\framebreak
	
	\vfill
	\begin{alertblock}{Kurzgesagt}
		Am Ende der Vorlesung sollt ihr mit den vorgestellten Konzepten der fortgeschrittenen Objektorientierung vertraut sein. Dies beinhaltet unter anderem das theoretische Verständnis der zugrundeliegenden Konzepte,
		sowie die \textbf{sprachunabhängige} Anwendung des gelernten. Ziel ist \textbf{nicht} das lernen von Java, sondern das übergreifende Verständnis, sodass das gelernte (theoretisch) in jeder Sprache angewandt werden kann!
	\end{alertblock}
	\vfill
\end{frame}

\begin{frame}[allowframebreaks]
	\frametitle{Termine \& Themen}
	
	\vfill
	\begin{tabularx}{\textwidth}{ll}
		\textbf{Termin} & \textbf{Geplante Themen}\\
		01.04.2019 14:00-17:15 & \\
		03.04.2019 14:00-17:15 & \\
		08.04.2019 14:00-17:15 & \\
		10.04.2019 14:00-17:15 & \\
		24.04.2019 14:00-17:15 & \\
		26.04.2019 14:00-17:15 & \\
		29.04.2019 14:00-17:15 & \\
		02.05.2019 14:00-17:15 & \\
	\end{tabularx}
	\vfill
	
	\framebreak
	
	\vfill
	\begin{tabularx}{\textwidth}{ll}
		\textbf{Termin} & \textbf{Geplante Themen}\\
		06.05.2019 14:00-17:15 & \\
		08.05.2019 14:00-17:15 & \\
		13.05.2019 14:00-17:15 & \\
		15.05.2019 14:00-17:15 & \\
		20.05.2019 14:00-17:15 & \\
		22.05.2019 14:00-17:15 & \\
		27.05.2019 14:00-17:15 & \\	
	\end{tabularx}
	\vfill
	
\end{frame}

\begin{frame}{Prüfungsleistung}{}
	\begin{itemize}
	\item \textbf{Keine} Klausur, denn:
		\begin{itemize}
			\item Schon 6 Klausuren dieses Semester
			\item Rein theoretische Überprüfungen für Programmieren sowieso eher fraglich
		\end{itemize}
	\item Stattdessen: \textbf{Portfolioprüfung}
		\begin{itemize}
			\item Besteht aus mehreren Teilleistungen (Hier: 3)
			\item Teil 1: Kurztest
			\item Teil 2\&3: Programmentwurf inklusive Dokumentation 
		\end{itemize}
	\end{itemize}
\end{frame}

\begin{frame}{Prüfungsleistung}{Portfolio}
	\begin{tabularx}{\textwidth}{|l|l|l|}
		\hline
		\textbf{Teilprüfung}&\textbf{Details}&\textbf{Gewichtung}\\
		\hline
		Kurztest&\makecell{Bearbeitungszeit: ~30 Minuten \\ Thema: Algorithmen} & \textbf{20\%}\\
		\hline
		Programmentwurf&\makecell{Bearbeitungszeit: ~4 Wochen \\ Thema: } & \textbf{50\%}\\
		\hline
		Dokumentation&Zum Programmentwurf&\textbf{30\%}\\
		\hline
	\end{tabularx}
\end{frame}