\begin{frame}{Allgemeines}{Über Arrays}
	\begin{itemize}
		\item Einfachste Listenstruktur
		\item Speichert Daten sequentiell im Speicher
		\item In der Regel nur in einer Dimension
		\item Aber auch "`mehrdimensionale"' Arrays möglich
		\item Logisch (und auch vom Zugriff) könnte man Arrays vergleichen mit:
		\begin{itemize}
			\item Vektoren bei eindimensionalen Arrays
			\item Matrizen für zwei- oder höherdimensionale Arrays
		\end{itemize}
	\end{itemize}
\end{frame}

\begin{frame}{Eigenschaften}{Von Arrays}
	\begin{itemize}
		\item Belegen einen fortlaufenden Bereich im Speicher
		\item Dadurch:
		\begin{itemize}
			\item Muss die Größe bei Initialisierung bekannt sein
			\item Sind Zugriffe auf die Elemente sehr schnell
			\item Löschen-/Einfügen von Elementen jedoch vergleichsweise "`teuer"'
		\end{itemize}
		\item Arrays sind meist für Datenmengen mit vielen Elementen(Ab ~65535) ungeeignet
		\item Warum?
	\end{itemize}
\end{frame}

\begin{frame}{Datenstruktur von Arrays}{Bildlich dargestellt}
%TODO: Abbildung Arrays
\end{frame}

\begin{frame}{Grundoperationen}{Zugriff auf Elemente}
	\begin{itemize}
		\item Zugriffe auf Listenelemente passieren immer in konstanter Zeit
		\item Dadurch ergibt sich die Komplexität zu: $O(1)$
		\item Begründung:
		\begin{itemize}
			\item Elemente liegen "`hintereinander"' im Speicher $\rightarrow$ Haben fortlaufende Speicheradressen
			\item Die Adresse des ersten Elements ist immer bekannt
			\item Heißt, bei Abrufen des $n$-ten Elements muss von der Startadresse nur eine bestimmte Schrittzahl addiert werden
			\item Jeder Zugriff auf ein Element ist somit (auf unterster Ebene) eine Addition und eine Leseoperation
		\end{itemize}
	\end{itemize}
\end{frame}


\begin{frame}{Graphische Darstellung}{Zugriff auf Elemente}
%TODO: Visuelle Darstellung Elementzugriff
\end{frame}

\begin{frame}{Grundoperationen}{An-/Einfügen von Elementen}
	\begin{itemize}
		\item Nachträgliches Einfügen von Elementen ist nicht trivial
		\item Grund dafür ist die Speicherstruktur von Arrays
		\item Dadurch, dass die Elemente fortlaufend im Speicher liegen...
		\begin{itemize}
			\item ...müsste bei Einfügen sichergestellt sein, dass der nachfolgende Speicher noch nicht genutzt ist (Was meist nicht der Fall ist)
			\item ...und alle Elemente die nach dem eingefügten Element liegen, nach rechts "`verschoben"' werden
		\end{itemize}
		\item Dadurch häufig das anlegen eines neuen Arrays nötig
		\item Und das ausführen von vielen Kopieroperationen
	\end{itemize}
\end{frame}

\begin{frame}{Einfügen in Arrays}{Visualisiert}
%TODO: Visualisierung einfügen
\end{frame}