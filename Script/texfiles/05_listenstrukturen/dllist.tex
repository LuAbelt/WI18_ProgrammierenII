\begin{frame}{Allgemeines}{Zu Double Linked Lists}
	\begin{itemize}
		\item Erweiterung der Linked List
		\item Gleiche Vorteile wie in dieser
		\item Jedes Element speichert nun Referenz auf Nachfolger (\texttt{next}) \textbf{und} Vorgänger(\texttt{previous})
		\item Dadurch verringerte Komplexität bei einigen Operationen
		\item Zusätzlich kennt die Liste das erste und letzte Element (\texttt{tail})
		\item Im ersten Element ist \texttt{previous} eine Referenz auf \texttt{null}
	\end{itemize}
\end{frame}

\begin{frame}{Double Linked Lists}{Visualisiert}
%TODO: Visualisierung Double Linked List
\end{frame}

\begin{frame}{Grundoperationen}{Anlegen der Liste}
	\begin{itemize}
		\item Double Linked List wird als leere Liste initialisiert
		\item \texttt{head} und \texttt{tail} werden als leere Liste intialisiert
		\item Sobald Elemente hinzugefügt werden, wird \texttt{head} und \texttt{tail} automatisch aktualisiert
		\item Speicher für Elemente wird beim hinzufügen reserviert
	\end{itemize}
\end{frame}

\begin{frame}{Grundoperationen}{Bestimmen der Größe}
	\begin{itemize}
		\item Größe bestimmen funktioniert theoretisch wie in der Linked List
		\item Wird jedoch in der Regel parallel gespeichert und verwaltet
		\begin{itemize}
			\item Sonst könnten einige Vorteile der Double Linked List nicht genutzt werden
			\item Da jedes mal erst die Größe bestimmt werden müsste
		\end{itemize}
	\end{itemize}
\end{frame}

\begin{frame}{Grundoperationen}{Zugriff auf Elemente}
	\begin{itemize}
		\item Ähnlich wie in der Linked List
		\item Jedoch kann die Anzahl der Operationen verringert werden
		\begin{itemize}
			\item Dadurch, dass wir von beiden Enden der Liste aus traversieren können
			\item Für Effizienzsteigerung muss jedoch Größe bekannt sein
		\end{itemize}
		\item Je nachdem, ob das Element im hinteren oder vorderen Teil der Liste ist
		\item Wird beginnend von hinten bzw. vorn traversiert um das Element abzurufen
		\item Dadurch sind theoretisch maximal $\frac{n}{2}$ Schritte nötig
		\item Da $O$ jedoch den konstanten Faktor $\frac{1}{2}$ ignoriert wird die Komplexität trotzdem mit $O(n)$ angegeben
	\end{itemize}
\end{frame}

\begin{frame}{Zugriff auf Elemente}{Visualisiert}
%TODO: Visualisierung Zugriff Elemente
\end{frame}

\begin{frame}{Grundoperationen}{Einfügen und Entfernen von Elementen}
	\begin{itemize}
		\item Operationen unterscheiden sich im Grunde kaum zur Linked List
		\item Nur, dass jetzt in den betroffenen Elementen mehr Referenzen geupdated werden müssen
		\item Zusätzlich zum \texttt{next} Element jetzt noch das \texttt{previous} Element
		\item Wie beim Zugriff verringert sich die Anzahl der Operationen, jedoch bleibt die Komplexität $O(N)$
	\end{itemize}
\end{frame}

\begin{frame}{Grundoperationen}{Konkatinieren von Listen}
	\begin{itemize}
		\item Funktioniert analog wie bei Linked Lists
		\item Lediglich wird noch die \texttt{previous} Referenz des Kopfes der zweiten Liste aktualisiert
		\begin{itemize}
			\item Auf das \texttt{tail} Element der ersten Liste
		\end{itemize}
		\item Verringerte Komplexität
		\begin{itemize}
			\item Dadurch, dass auf \texttt{tail} Element direkt zugegriffen werden kann
			\item Komplexität ist somit konstant: $O(1)$
			\item Teilweise speichern auch normale Linked Lists die \texttt{tail} Referenz für verringerte Komplexität beim konkatinieren
		\end{itemize}
	\end{itemize}
\end{frame}