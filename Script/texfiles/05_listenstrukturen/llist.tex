\begin{frame}{Allgemeines}{Über Linked Lists}
	\begin{itemize}
		\item Simple Listenstruktur
		\item Elemente können verteilt im Speicher liegen
		\item Jedes Element speichert einen Datenwert
		\item ...und eine Referenz zum Nachfolger
		\item Speichern mehrdimensionaler Daten theoretisch möglich
		\begin{itemize}
			\item Wenn der Datenwert auch wieder eine Linked List ist
			\item Zugriff allerdings weniger intuitiv als im Array
		\end{itemize}
	\end{itemize}
\end{frame}


\begin{frame}{Eigenschaften}{Von Linked Lists}
	\begin{itemize}
		\item Muss keinen fortlaufenden Teil im Speicher belegen
		\item Dadurch entfallen einige Nachteile des Arrays:
		\begin{itemize}
			\item Größe muss nicht bei Initialisierung bekannt sein
			\item Liste kann dynamisch wachsen bzw. schrumpfen (Ohne "`teuere"' Kopiervorgänge)
			\item Einfügen bzw. Entfernen ist schneller
		\end{itemize}
		\item Allerdings ist der Zugriff auf Elemente langsamer
	\end{itemize}
\end{frame}

\begin{frame}{Datenstruktur von Linked Lists}{Listenelemente}
	\begin{itemize}
		\item Jedes Element einer Linked List ist ein Objekt
		\item Jedes Objekt vom Typ \textit{ListElement} besteht aus:
		\begin{itemize}
			\item Einem Member das den Wert des aktuellen Elements speichert. Datentyp ist der Typ, den die Liste speichert (\texttt{Double}, \texttt{Integer} etc.)
			\item Einem Member welches das nächste Element (bzw. eine Referenz darauf) in der Liste repräsentiert. Datentyp ist hier \texttt{ListElement}
		\end{itemize}
		\item Gibt es kein nächstes Element in der Liste kann dies über ein spezielles \texttt{tail} Objekt oder eine \texttt{null} Referenz dargestellt werden
	\end{itemize}
\end{frame}

\begin{frame}{Datenstruktur von Linked Lists}{Liste}
	\begin{itemize}
		\item Die Liste vom Typ \texttt{LinkedList} besteht an sich lediglich aus:
		\begin{itemize}
			\item Einem Member \texttt{head}(Auch: \texttt{first}, \texttt{top} o.Ä.) vom Typ \texttt{ListElement}, welches das erste Element der Liste repräsentiert.
			\item Den für die Liste benötigten Operationen zum Zugriff und Manipulation von Listenelementen
		\end{itemize}
		\item Sollte die Liste leer sein, so ist \texttt{head} ein Verweis auf \texttt{null} oder ein spezielles Element, das das Ende einer Liste repräsentiert.
	\end{itemize}
\end{frame}

\begin{frame}{Datenstruktur von Linked Lists}{Visualisiert}
%TODO: Visualisierung LinkedLists
\end{frame}

\begin{frame}{Grundoperationen}{Anlegen einer Linked List}
	\begin{itemize}
		\item Bei Anlegen einer neuen Linked List wird diese als leere Liste initialisiert
		\item Anders als beim Array muss kein Speicher für die Listenelemente reserviert werden
		\item Das heißt, das \texttt{head} Element wird mit einer \texttt{null} Referenz initialisiert
		\item Das erste Element, das hinzugefügt wird, wird automatisch zum \texttt{head} Elemente der Liste
		\item Wird ein Element hinzugefügt, wird erst zum Zeitpunkt des Hinzufügens der Speicher für dieses Element reserviert
	\end{itemize}
\end{frame}

\begin{frame}{Grundoperationen}{Zugriff auf Elemente}
	\begin{itemize}
		\item Zugriff auf Elemente muss immer sequentiell vom \texttt{head} Element aus erfolgen
		\item Zugriff auf das $n$-te Element erfolgt durch wiederholtes Zugreifen auf das \texttt{next} Element
		\item Bedeutet:
		\begin{itemize}
			\item Je weiter hinten das gesuchte Element steht, desto mehr Operationen sind möglich
			\item Somit ergibt sich die Komplexität zu $O(N)$
		\end{itemize}
	\end{itemize}
\end{frame}

\begin{frame}[fragile]{Zugriff auf Listenelemente}{Mögliche Codeimplementierung}
\lstset{style=java}
\begin{lstlisting}
public T getElement(int index){
	ListElement res = head;
	for(int i=0;i<index;i++){
		res = res.getNext();
	}
	return res.getValue();
}
\end{lstlisting}
\end{frame}