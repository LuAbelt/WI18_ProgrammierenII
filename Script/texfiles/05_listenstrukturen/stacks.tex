\begin{frame}{Allgemeines}{Zu Stacks}
	\begin{itemize}
		\item "`Stapel"' $\rightarrow$ Beschreibt Funktionsweise sehr passend
		\item Listenstruktur in der nur Zugriff auf das "`oberste"' Element möglich ist
		\item Oberstes Element ist hier das zuletzt hinzugefügte
		\item Daten werden intern (als private Variable) verwaltet
		\begin{itemize}
			\item z.B. als Array oder Linked List
		\end{itemize}
	\end{itemize}
\end{frame}

\begin{frame}{Allgemeines}{Formale Definition Stack (Vgl. \cite{fahr:list})}
	\begin{alertblock}{Definition Stack}
	Ein Stack lässt sich \textbf{rekursiv} definieren. Zu jedem Zeitpunkt ist ein Stack entweder:
	\begin{itemize}
		\item Eine leere Datenstruktur
		\item Bestehend aus einem obersten Element und dem Rest, wobei der Rest wieder ein Stack ist
	\end{itemize}
	\end{alertblock}
\end{frame}

\begin{frame}{Allgemeines}{Stack Interface (Vgl. \cite{stacksqueues})}
	\begin{itemize}
		\item Grundlegendes Interface besteht aus zwei Operationen:
		\begin{itemize}
			\item \texttt{push} - Fügt ein Element hinten an den Stack an
			\item \texttt{pop} - Gibt das oberste Elemente des Stacks zurück und entfernt dieses aus dem Stack
		\end{itemize}
		\item Außerdem in der Regel noch Teil des Interfaces:
		\begin{itemize}
			\item \texttt{peek} - Gibt das oberste Element zurück, ohne es zu entfernen
			\item \texttt{isEmpty}
		\end{itemize}
	\end{itemize}
\end{frame}

\begin{frame}{Grundoperationen}{Anlegen eines Stacks}
	\begin{itemize}
		\item Stack wird als leere Datenstruktur initialisiert
		\item Je nachdem wie die Daten intern strukturiert sind muss ggf. eine Größe initialisiert werden
		\item Sobald das erste Element hinzugefügt wird, wird dieses zum obersten (\texttt{top}) Element
	\end{itemize}
\end{frame}

\begin{frame}{Grundoperationen}{Zugriff auf Elemente}
	\begin{itemize}
		\item Zugriff nur auf das letzte Element der Liste (Zletzt hinzugefügt)
		\item \texttt{pop} gibt das letzte Element zurück und entfernt dieses aus der Liste
		\item \texttt{peek} Gibt das letzte Element zurück ohne es zu entfernen
		\begin{itemize}
			\item Wiederholte Zugriffe auf \texttt{peek} geben also immer das gleiche Ergebnis zurück (Sofern der Stack zwischenzeitlich nicht manipuliert wurde)
		\end{itemize}
		\item Zugriff auf beliebige Elemente im Stack nicht möglich (Zuminndest sollte der Stack dafür kein Interface anbieten)
	\end{itemize}
\end{frame}

\begin{frame}{Grundoperationen}{An-/Einfügen von Elementen}
	\begin{itemize}
		\item Hinzufügen nur für das Ende des Stacks vorgesehen
		\item Dafür ist die Komplexität $O(1)$
		\item Hinzufügen in der Mitte des Stacks nicht vorgesehen
		\begin{itemize}
			\item Daher sehr aufwändig
			\item Es müssen von hinten sukzessiv Elemente entfernt werden, bis der gewünschte Index die letzte Position des Stacks wäre
			\item Dann wird das neue Element hinzugefügt
			\item Dann müssen alle zuvor entfernten Elemente wieder hinzugefügt werden
			\item Diese Operation sollte aber \textbf{nicht} Teil des Stack Interfaces sein
		\end{itemize}
	\end{itemize}
\end{frame}

\begin{frame}{Grundoperationen}{Bestimmen der Größe}
	\begin{itemize}
		\item Größe des Stacks wird in der Regel parallel verwaltet
		\item Und bei Aufrufen von \texttt{push} und \texttt{pop} aktualisiert
		\item Sonst wäre zum zählen folgendes Vorgehen nötig:
		\begin{itemize}
			\item Kopiere den Stack
			\item Entferne aus dem kopierten Stack solange Elemente bis der Stack leer ist
			\item Bei jedem entfernen erhöhe einen Zähler um 1
		\end{itemize}
	\end{itemize}
\end{frame}

\begin{frame}{Grundoperationen}{Entfernen von Elementen}
	\begin{itemize}
		\item Stack Interface bietet nur Möglichkeit, letztes Element zu entfernen
		\item Über die \texttt{pop} Funktion
		\item Entfernen in der Mitte vom Stack nicht möglich
		\begin{itemize}
			\item Kann theoretisch durch einen Benutzer realisiert werden (Ähnliches Vorgehen wie beim Einfügen in der Mitte)
			\item Sollte jedoch nicht Teil des Interfaces sein
		\end{itemize}
	\end{itemize}
\end{frame}

\begin{frame}{Grundoperationen}{Konkatinieren von Stacks}
	\begin{itemize}
		\item Kombinieren von zwei Stack relativ aufwendig
		\item Wenn das Ende des zweiten Stacks auch das Ende des zusammengefügten Stacks sein sollte
		\item Denn: Bei entfernen aller Elemente vom Stack wird die Reihenfolge der Elemente invertiert
		\item Nötiges Vorgehen:
		\begin{itemize}
			\item Entferne über \texttt{pop} sukzessive die Elemente vom zweiten Stack bis dieser leer ist und füge sie (über \texttt{push}) zu einem temporären Stack hinzu
			\item Entferne sukzessive die Elemente vom temporären Stack und füge sie an den ersten Stack an
		\end{itemize}
	\end{itemize}
\end{frame}

\begin{frame}{Anwendungsfälle}{Für Stacks}
	\begin{itemize}
		\item Backtracking Algorithmen - Jeder Schritt wird auf dem Stack gespeichert und kann über den Stack "`zurückgegangen werden"'
		\begin{itemize}
			\item Beispiel: Labyrinth Algorithmus.
			\item Jeder Schritt wird auf einem Stack gespeichert
			\item Bei erreichen einer Sackgasse werden die zuletzt gemachten Schritte so lange zurück gegangen, bis ein anderer Schritt möglich ist
		\end{itemize}
		\item Rückgängig Funktion - Jede Aktion wird auf einem Stack abgelegt und kann somit rückgängig gemacht werden
	\end{itemize}
\end{frame}