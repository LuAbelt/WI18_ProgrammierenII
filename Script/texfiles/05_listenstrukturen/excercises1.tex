\begin{frame}{Aufgabe 1}{ArrayList}
Im ersten Teil haben wir über die Eigenschaften von Arrays gesprochen. Unter anderem ging es darum, 
dass Arrays durch ihre Struktur relativ unflexibel sind, was das hinzufügen von Elementen betrifft.

Das \texttt{Collections} Interface bietet mit der \texttt{ArrayList} Klasse eine Liste an, die im Hintergrund
ein Array zum speichern von Daten anbietet. Diese Liste bietet die zwei Funktionen \texttt{size()} und \texttt{capacity} an.
\begin{enumerate}
	\item Untersucht den Unterschied zwischen den beiden Funktionen
	\item Untersucht, wie sich die Ergebnisse der beiden Funktionen verhalten, wenn Elemente hinzugefügt und entfernt werden
	\item Welche Rückschlüsse lassen sich daraus über die Datenorganisation der ArrayList treffen?
\end{enumerate}
\end{frame}

\begin{frame}{Aufgabe 2}{Linked List}
	\begin{enumerate}
		\item Entwerft eine \texttt{LinkedListElement<T>} Klasse mit den in der Vorlesung vorgestellten Eigenschaften
		\item Entwerft eine \texttt{LinkedList<T>} Klasse, die zur Speicherung der Daten die \texttt{LinkedListElement<T>} Klasse verwendet. Die Liste sollte folgende Funktionen integrieren:
		\begin{itemize}
			\item \texttt{get(int index)} - Gibt das Element an einem bestimmten Index zurück
			\item \texttt{size()} - Gibt die Anzahl der Elemente in der Liste zurück
			\item \texttt{insert} - Fügt ein bestimmtes Element an einem gegebenen Index ein
			\item \texttt{remove(index i)} - Entfernt das Element an dem gegebenen Index
		\end{itemize}
	\end{enumerate}
\end{frame}

\begin{frame}{Aufgabe 3}{Double Linked List}
Entwerft analog zu Aufgabe 2 die Klasse \texttt{DoubleLinkedListElement<T>} und \texttt{DoubleLinkedList<T>}
\end{frame}