\begin{frame}{Allgemeines}{Zu Listenstrukturen}
	\begin{itemize}
		\item Daten können verschieden strukturiert werden
		\begin{itemize}
			\item Abhängig vom Inhalt
			\item Oder Verwendungszweck
		\end{itemize}
		\item Algorithmen sind oft von der darunterliegenden Datenstruktur abhängig
		\item Je nach Problem sind verschiedene Datenstrukturen besser oder schlechter geeignet
		\item Wie wir unsere Daten strukturieren hängt also ab von:
		\begin{itemize}
			\item Der Beschaffenheit der Daten
			\item Wofür werden die Daten verwendet
		\end{itemize}
	\end{itemize}
\end{frame}

\begin{frame}{Allgemeines}{Praktische Umsetzung}
	\begin{itemize}
		\item Listenstrukturen sind als "`ready to use "' Strukturen in allen gängigen Sprachen vorhanden:
		\begin{itemize}
			\item \textit{Collections} Framework in Java
			\item \textit{Collection} Interface in C\#
			\item \textit{STL-Container} in C++
		\end{itemize}
		\item Die grundlegenden Eigenschaften haben wir schon besprochen...
		\item ...jetzt gehen wir etwas genauer auf die technischen Hintergründe ein
	\end{itemize}
\end{frame}

\begin{frame}{Ziele des Moduls}
	\begin{itemize}
		\item Am Ende des Moduls sollt ihr:
		\begin{itemize}
			\item Die Unterschiede der einzelnen Listenstrukturen kennen
			\item Die Vor- und Nachteile benennen können
			\item Die Grundoperationen von Listenstrukturen kennen
			\item Die Unterschiede der Grundoperationen in den Listenstrukturen verstehen
		\end{itemize}
	\end{itemize}
\end{frame}

\begin{frame}{Grundoperationen}{In Listenstrukturen}
	\begin{itemize}
		\item Für Listen existieren die folgenden Grundoperationen:
		\begin{itemize}
			\item Anlegen der Listenstruktur
			\item Zugriff auf ein Listenelement
			\item An-/Einfügen von Listenelementen
			\item Ermitteln der Länge der Liste(Bzw. bestimmen der Anzahl der Elemente)
			\item Entfernen von Elementen
			\item Konkatinieren von Listen
		\end{itemize}
	\end{itemize}
\end{frame}