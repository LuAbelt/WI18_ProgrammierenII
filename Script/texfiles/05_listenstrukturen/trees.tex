\begin{frame}{Allgemeines}{Baumstrukturen}
	\begin{itemize}
		\item Unterscheiden sich von den bisherigen Strukturen
		\item Bisher hatte jedes Element einer Datenstruktur immer einen "`Nachfolger"'
		\item ...und einen Vorgänger
		\item Bäume können jedoch mehrere "`Nachfolger"' haben
		\begin{itemize}
			\item Man spricht in der Regel von \textbf{Knoten}
			\item Es gibt genau einen Knoten im Baum, der keinen Eingang ("`Vorgänger"') besitzt
			\begin{itemize}
				\item Dies ist der \textbf{Wurzelknoten} des Baumes
			\end{itemize}
			\item Alle anderen Knoten haben genau einen Eingang
		\end{itemize}
		\item In der Regel beschäftigt man sich hauptsächlich mit \textit{Binärbäumen}
		\begin{itemize}
			\item Diese haben \textbf{zwei} Nachfolger
		\end{itemize}
	\end{itemize}
\end{frame}

\begin{frame}{Allgemeines}{Definition von Bäumen}
	\begin{alertblock}{Graphentheoretische Definition}
	Ein Baum ist ein endlicher, schwach zusammenhängender gerichteter Graph, für dessen Knotenpunkte gilt:
	\begin{enumerate}
		\item Es gibt genau einen Knoten, der keinen Eingang hat (Wurzel des Baumes)
		\item Alle übrigen Knotenpunkte haben genau einen Eingang.
	\end{enumerate}
	\end{alertblock}
	
	\begin{alertblock}{Rekursive Definition(Binärbaum)}
	Eine Baumstruktur vom Grundtyp T ist:
	\begin{enumerate}
		\item Eine leere Struktur
		\item Ein Knoten vom Typ T mit genau zwei disjunkten Teilbäume vom Grundtyp T
	\end{enumerate}
	\end{alertblock}
\end{frame}

\begin{frame}{Allgemeines}{Struktur der Knoten im Baum}
	\begin{itemize}
		\item Grundsätzlich besteht ein Knoten(Im Binärbaum) aus drei Elementen
		\begin{itemize}
			\item Einem Schlüssel (=Datenwert) für den Knoten
			\item Einen linken Teilbaum
			\item Einen rechten Teilbaum
		\end{itemize}
	\end{itemize}
\end{frame}

\begin{frame}{Eigenschaften}{Von Binärbäumen}
	\begin{alertblock}{Vollständige Binärbäume}
	In einem Binärbaum kann jede Ebene maximal $2^{(Tiefe-1)}$ Knoten besitzen. Ein Binärbaum ist dann vollständig ausgeglichen, wenn auf jeder Ebene, mit Ausnahme der letzten, genau diese
	$2^{(Tiefe-1)} Knoten existieren und die Knoten auf der letzten Ebene soweit links wie möglich stehen
	\end{alertblock}
	
	\begin{alertblock}{Suchbaum}
	Wenn für alle Knoten K eines Baumes gilt, dass...
	\begin{enumerate}
		\item ...alle Schlüssel im linken Teilbaum von K kleiner...
		\item ...alle Schlüssel im rechten Teilbaum von K größer...
	\end{enumerate}
	... als der Schlüssel des Knotens K sind, so handelt es sich um einen Suchbaum
	\end{alertblock}
\end{frame}

\begin{frame}{Grundoperationen}{In Binärbäumen}
	\begin{itemize}
		\item Zur Vereinfachung betrachten wir für Binärbäume (Suchbäume) folgende Operationen:
		\begin{itemize}
			\item Einfügen von Elementen
			\item Entfernen von Elementen
			\item Bestimmen der Anzahl von Elementen
		\end{itemize}
		\item Grundoperationen basieren fast immer auf rekursivem Aufruf auf den Teilbäumen!
	\end{itemize}
\end{frame}

\begin{frame}{Grundoperationen}{Einfügen von Elementen}
	\begin{itemize}
		\item Für das Einfügen eines Elements $X$ in einen Baum $B$ werden vier Fälle unterschieden:
		\begin{itemize}
			\item Ist $B$ leer, so ist das Ergebnis der Baum mit dem Schlüssel $X$
			\item Ist $X$ identisch mit dem Schlüssel von $B$ so ist $X$ bereits im Baum und wird nicht hinzugefügt
			\item Ist $X$ kleiner als der Schlüssel von $B$ so füge $X$ im linken Teilbaum ein
			\item Ist $X$ größer als der Schlüssel von $B$ so füge $X$ im rechten Teilbaum ein
		\end{itemize}
	\end{itemize}
\end{frame}

\begin{frame}{Einfügen in Suchbäume}{Beispiele}
%TODO: Beispiele Einfügen
\end{frame}

\begin{frame}{Grundoperationen}{Entfernen aus Suchbäumen}
\begin{itemize}
	\item Beim Entfernen eines Elements mit dem Schlüssel $X$ können vier Fälle auftreten:
	\begin{itemize}
		\item Der Baum enthält $X$ nicht - Fertig.
		\item Der Knoten mit dem Schlüssel $X$ hat keinen Nachfolger - $X$ wird entfernt, Fertig.
		\item Der Knoten mit dem Schlüssel $X$ hat genau einen Nachfolger
		\item Der Knoten mit dem Schlüssel $X$ hat genau zwei Nachfolger
	\end{itemize}
\end{itemize}
\end{frame}

\begin{frame}{Entfernen}{$X$ hat einen Nachfolger}
	\begin{itemize}
		\item Simpler Fall
		\item Hier "`rutscht"' der Nachfolger (Gesamter Teilbaum) einfach an die Stelle des zu entfernenden Elements
		\item Keine weiteren Operationen - Fertig.
	\end{itemize}
\end{frame}

\begin{frame}{Entfernen}{Visualisiert für einen Nachfolger}
%TODO: Visualisierung Entfernen
\end{frame}

\begin{frame}{Entfernen}{$X$ hat zwei Nachfolger}
\end{frame}