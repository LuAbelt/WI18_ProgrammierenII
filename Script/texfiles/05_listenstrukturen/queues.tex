\begin{frame}{Allgemeines}{Zu Queues (Vgl. \cite{stacksqueues})}
	\begin{itemize}
		\item Funktionieren ähnlich wie Stacks
		\item Hier jedoch andere Zugriff:
		\begin{itemize}
			\item Elemente können nur hinten angefügt werden
			\item ...und vorn entfernt werden
		\end{itemize}
		\item Stellt eine "`Warteschlange"' dar
		\item Interne Datenstruktur wie beim Stack variabel
		\item Wichtig ist, dass von außen nur über das Queue Interface zugegriffen werden kann
	\end{itemize}
\end{frame}

\begin{frame}{Allgemeines}{Queue Interface (Vgl. \cite{stacksqueues})}
	\begin{itemize}
		\item Queue Interface besteht aus zwei grundlegenden Operationen:
		\begin{itemize}
			\item \texttt{enqueue} - Hängt ein Element hinten an die Queue an 
			\item \texttt{dequeue} - Gibt das ersten Element der Queue zurück und entfernt dieses
		\end{itemize}
		\item Wie im Stack gibt es meist noch weitere Hilfsfunktionen wie \texttt{isEmpty} oder \texttt{peek}
	\end{itemize}
\end{frame}

\begin{frame}{Grundoperationen}{An-/Einfügen von Elementen}
	\begin{itemize}
		\item Anfügen von Elementen nur am Ende möglich
		\item Einfügen in der Mitte nicht vorgesehen (In der Standardqueue)
		\item Um ein Element an der Stelle $i$ einzufügen müsste man:
		\begin{itemize}
			\item Die Größe der Queue bestimmen ($n$)
			\item Die ersten $i-1$ Elemente der Queue entfernen und direkt hinten anfügen
			\item Dann das einzufügende Element anhängen
			\item Dann die vordersten $n-i+1$ Elemente entfernen und direkt hinten anfügen
		\end{itemize}
		\item Solche Operationen sind jedoch nicht Teil des Standardinterfaces
	\end{itemize}
\end{frame}

\begin{frame}{Grundoperationen}{An-/Einfügen von Elementen}
	\begin{itemize}
		\item Es gibt Queues in denen neue Element nicht zwangsweise hinten angehängt werden
		\item Sondern nach anderen Kriterien in der Queue sortiert werden, z.B.
		\begin{itemize}
			\item Angegebener Priorität
			\item (Hash-)Wert des Elements
		\end{itemize}
		\item Wichtig hierbei: Das Interface sollte sich hierbei nach außen nicht maßgeblich verändern
		\item Die Logik wie das Element in die interne Struktur eingefügt wird, bleibt intern!
	\end{itemize}
\end{frame}

\begin{frame}{Grundoperationen}{Bestimmen der Größe}
	\begin{itemize}
		\item Wie auch im Stack ist die Größenbestimmung sehr aufwendig
		\item Kopieren der Queue und sukzessives entfernen notwendig
		\item Daher wird auch hier die Größe meist "`manuell"' verwaltet und beim Hinzufügen/Entfernen aktualisiert
	\end{itemize}
\end{frame}

\begin{frame}{Grundoperationen}{Entfernen von Elementen}
	\begin{itemize}
		\item Das Interface erlaubt nur das entfernen des vordersten Elements
		\item Um Elemente in der MItte zu entfernen müsste ähnlich vorgegangen werden wie beim Einfügen
		\item Auch hier gilt wieder: Das Entfernen "`in der Mitte"' sollte nicht Teil des Interfaces sein
	\end{itemize}
\end{frame}

\begin{frame}{Grundoperationen}{Konkatinieren von Queues}
	\begin{itemize}
		\item Kombinieren von Queues ist - vergleichsweise - "`simpel"'
		\begin{itemize}
			\item Im Grunde muss nur wiederholt das erste Element der zweiten Queue entfernt werden
			\item ...und direkt an die erste angefügt werden
		\end{itemize}
		\item Je nachdem wie die Queue intern strukturiert ist, kann dies über eine bereitgestellte "`append"' Funktion optimiert werden
		\begin{itemize}
			\item Zum Beispiel durch direktes zusammenfügen der internen Linked Lists
		\end{itemize}
	\end{itemize}
\end{frame}

\begin{frame}{Anwendungsfälle}{Für Queues}
	\begin{itemize}
		\item Scheduling von nacheinander ablaufenden Operationen
		\item Planung von Prozessen in gemeinsam genutzten Ressourcen
		\begin{itemize}
			\item Wenn eine Ressource gerade belegt ist und ein weterer Prozess sie aber benötigt
			\item Wird der Prozess zur nächsten Berechnung in der Queue gespeichert
		\end{itemize}
	\end{itemize}
\end{frame}