\begin{frame}{Datentypen}{Unterscheidung}
    \begin{itemize}
        \item Primitive Datentypen
        \item Objektorientierte Datentypen
        \item Strukturierte Datentypen
    \end{itemize}
\end{frame}

\begin{frame}{Primitive Datentypen}{Beispiele}
    \begin{itemize}
    \item Beispiele:
        \begin{itemize}
            \item int
            \item boolean
            \item char
            \item float
            \item double
            \item long
            \item byte
        \end{itemize}
    \end{itemize}
\end{frame}

\begin{frame}{Primitive Datentypen}{Eigenschaften}
    \begin{itemize}
        \item Rein zum Speichern von Daten
        \item Feste Speichergröße
        \item Feste Präzision $\Rightarrow$ Diskret
        \item Fest definierte Ober- und Untergrenze
        \item Implementieren selbst keine Algorithmen, sondern nur simple Operationen
        \item Menge an Operationen beschränkt (+,-,*,/ und ggf. Bitwise Operations)
        \begin{itemize}
            \item Und nicht erweiterbar (Vgl. Objektorientierte Datentypen)
        \end{itemize}
    \end{itemize}
    Vgl. \cite{wiki:primdt}, \cite{orac:primdt}
\end{frame}

\begin{frame}{Primitive Datentypen}{Eigenschaften (In Java)}
    \begin{tabular}{|c|c|c|c|}
    \hline
    \textbf{Typ} & \textbf{Größe (Bits)} & \textbf{Minimum} & \textbf{Maximum}
    \\
    \hline
    \visible<+->{\textbf{byte}} & \visible<+->{8} & \visible<+->{$ -128 $} & \visible<+->{$ 127 $}\\
    \hline
    \textbf{char} & \visible<+->{16} & \visible<+->{$ 0 $} & \visible<+->{$ 2^{16}-1 $}\\
    \hline
    \textbf{short} & \visible<+->{16} & \visible<+->{$ -2^{15} $} & \visible<+->{$ 2^{15}-1 $}\\
    \hline
    \textbf{int} & \visible<+->{32} & \visible<+->{$ -2^{31} $} & \visible<+->{$ 2^{31}-1 $}\\
    \hline
    \textbf{long} & \visible<+->{64} & \visible<+->{$ -2^{63} $} & \visible<+->{$ 2^{63}-1 $}\\
    \hline
    \textbf{float} & \visible<+->{32} & \visible<+->{$ \pm 1.4\text{E-}45 $} & \visible<+->{$ \pm 3.4\text{E+}38 $}\\
    \hline
    \textbf{double} & \visible<+->{64} & \visible<+->{$ \pm 4.9\text{E-}324 $} & \visible<+->{$ \pm 1.7\text{E+}324 $}\\
    \hline
    \textbf{boolean} & \visible<+->{Undefiniert} & \multicolumn{2}{c|}{\visible<+->{Nur \tt{true} und \tt{false}}}\\
    \hline
    \end{tabular}
    
    
    Vgl. \cite{wiki:primdt}
\end{frame}

\begin{frame}{Primitive Datetypen}{Umwandlung}
    \begin{alertblock}{Implizite Umwandlung}
        Für die meisten Fälle gilt: Primitive Datentypen lassen sich grundsätzlich implizit umwandeln, wenn die Präzision des Zieldatentyps höher ist als die des Ursprungsdatentyps. Also:
        
        \begin{center}
        \texttt{byte} $\Rightarrow$ \texttt{short} $\Rightarrow$ \texttt{int} $\Rightarrow$ \texttt{long} $\Rightarrow$ \texttt{float} $\Rightarrow$ \texttt{double}
        \end{center}
    \end{alertblock}
    Vgl. \cite{wiki:primdt}
\end{frame}

\begin{frame}{Primitive Datentypen}{Umwandlung}
    \begin{alertblock}{Explizite Umwandlung}
        Alle anderen Umwandlungen müssen explizit durchgeführt werden zum Beispiel durch nutzen von Casting.
        
        \textbf{Achtung:} Dies kann zu Datenverlust führen. Beispiel: Umwandlung von \texttt{float} in \texttt{int} $\Rightarrow$ Dezimalstellen gehen verloren.
    \end{alertblock}
    Vgl. \cite{wiki:primdt}
\end{frame}

\begin{frame}{Objektorientierte Datentypen}{Grundlegendes}
    \begin{itemize}
        \item Zusammenfassen von Daten und Algorithmen in eine gemeinsame Instanz $\Rightarrow$ \textbf{Klasse}
        \item Von einer Klasse können mehrere unabhängige Instanzen existieren $\Rightarrow$ \textbf{Objekt}
        \item Entwickler entscheidet wie mit der Klasse interagiert werden muss $\Rightarrow$ \textbf{Datenkapselung}
        \item Klassen lassen sich wiederverwenden $\Rightarrow$ Verringerter Arbeitsaufwand durch gut designete Klassen
        \begin{itemize}
            \item Durch Verwendung weiterführender Konzepte (wie z.B. Design Patterns) noch gestärkt
        \end{itemize}
        \item Bilden die Grundlage von Java
    \end{itemize}
\end{frame}

\begin{frame}{Objektorientierte Datentypen}{Vererbung}
    \begin{itemize}
        \item Grundpfeiler der Objektorientierung
        \item Unterklassen \textbf{müssen} die Schnittstellen der Superklasse übernehmen
        \begin{itemize}
            \item Können neue Methoden und Variablen hinzufügen
            \item Können vorhandene Methoden überschreiben $\Rightarrow$ Implementieren spezifischer Logik
            \item Variablen bzw. Methoden können durch Vererbung \textbf{NICHT} entfernt werden
        \end{itemize}
        \item Objekte der Unterklassen können anstelle der Oberklasse treten
        \begin{itemize}
            \item Frage: Warum nicht umgekehrt?
        \end{itemize}
    \end{itemize}
\end{frame}

%TODO: Hier evtl. Codebeispiel Vererbung?

\begin{frame}{Objektorientierte Datentypen}{Vererbung}
    \begin{itemize}
        \item Formales OOP erlaubt Mehrfachvererbung, nutzbar in z.B.:
        \begin{itemize}
            \item C++
            \item Python
            \item Perl
        \end{itemize}
        \item In Java: Nur Einfachvererbung erlaubt
        \begin{itemize}
            \item Weitere Sprachen mit Einfachvererbung:
            \item C\#
            \item Ruby
        \end{itemize}
        \item Einige Funktionalitäten lassen sich über Interfaces abbilden
    \end{itemize}
\end{frame}

\begin{frame}{Objektorientierte Datentypen}{Interfaces}
    \begin{minipage}{0.4\textwidth}
            \begin{figure}
                \includegraphics[height=3.35cm]{graph/judt}
                \caption*{Quelle: \cite{pic:judt}}
                \label{fig:ll}
            \end{figure}
        \end{minipage}
        \hfill
        \begin{minipage}{0.55\textwidth}
            \textit{„Java hat das Konzept der Mehrfachvererbung über die Nutzung von Interfaces durch die Hintertür eingeführt.“} \\\\Andreas Judt
        \end{minipage}
\end{frame}

\begin{frame}{Objektorientierte Datentypen}{Interfaces}
    \begin{itemize}
        \item Definieren einen Satz an Methoden
        \item Implementierung der Funktionen erfolgt auf Klassenebene
        \begin{itemize}
            \item Seit Java 8: Definition von Standardimplementierung im Interface möglich
        \end{itemize}
        \item Klassen implementieren beliebig viele Interfaces
        \item Java bietet viele Standardinterfaces wie z.B.:
        \begin{itemize}
            \item Comparable
            \item Iterable
            \item Serializable
        \end{itemize}
    \end{itemize}
\end{frame}

\begin{frame}{Strukturierte Datentypen}{Allgemeines}
    \begin{itemize}
        \item Sind irgendwo zwischen Primitiven und Objektorientierten Datentypen einzuordnen
        \item In Java jedoch als Klassen umgesetzt (Sofern vorhanden)
        \item Wir betrachten folgende:
        \begin{itemize}
            \item Enumerations
            \item Structs
            \item Union
        \end{itemize}
    \end{itemize}
\end{frame}

\begin{frame}{Strukturierte Datentypen}{Enumerations}
    \begin{itemize}
        \item Ist eine Menge von geordneten, benannten Werten
        \item Werte der Enumeration können über Namen geprüft werden
        \item Anwendung:
        \begin{itemize}
            \item Beschränken einer Variable auf bestimmte definierte Werte
            \item Vermeiden von „Magic Numbers“ $\Rightarrow$ Code-Lesbarkeit
        \end{itemize}
        \item In den meisten Sprachen sind Enum Werte im Hintergrund nur Integer Konstanten
        \item In Java: Automatisch Klasse
        \begin{itemize}
            \item Dadurch höhere Flexibilität
        \end{itemize}
    \end{itemize}
Vgl. \cite{w3:enum}
\end{frame}

\begin{frame}[fragile]{Enumerations}{Beispiel I}
Enumeration zur Repräsentation der Wochentage sieht zum Beispiel so aus:
\lstset{style=java}
\begin{lstlisting}
enum Weekday {
    MONDAY,
    TUESDAY,
    WEDNESDAY,
    THURSDAY,
    FRIDAY,
    SATURDAY,
    SUNDAY
}
//Spaetere Nutzung im Code:
Weekday myDay = Weekday.TUESDAY;
\end{lstlisting}
Siehe \cite{w3:enum}
\end{frame}

\begin{frame}{Enumerations}{Beispiel II}
	\begin{itemize}
		\item Durch die Klassenstruktur können Enumerations in Java komplexer sein
		\item Definition von mehreren Variablen der Enum-Werte
		\item Definition von Konstruktoren
		\item Definition von Methoden
	\end{itemize}
    
\end{frame}

\begin{frame}[fragile]{Enumerations}{Komplexes Beispiel I}
\lstset{style=java}
\begin{lstlisting}
public enum Planet {
    MERCURY (3.303e+23, 2.4397e6),
    VENUS   (4.869e+24, 6.0518e6),
    EARTH   (5.976e+24, 6.37814e6),
    MARS    (6.421e+23, 3.3972e6),
    JUPITER (1.9e+27,   7.1492e7),
    SATURN  (5.688e+26, 6.0268e7),
    URANUS  (8.686e+25, 2.5559e7),
    NEPTUNE (1.024e+26, 2.4746e7);

    private final double mass;   // in kilograms
    private final double radius; // in meters
\end{lstlisting}
\end{frame}

\begin{frame}[fragile]{Enumerations}{Komplexes Beispiel II}
\lstset{style=java}
\begin{lstlisting}[firstnumber=13]
    Planet(double mass, double radius) {
        this.mass = mass;
        this.radius = radius;
    }
    private double mass() { return mass; }
    private double radius() { return radius; }

    // universal gravitational constant  (m3 kg-1 s-2)
    public static final double G = 6.67300E-11;

    double surfaceGravity() {
        return G * mass / (radius * radius);
    }
\end{lstlisting}
\end{frame}

\begin{frame}[fragile]{Enumerations}{Komplexes Beispiel II}
\lstset{style=java}
\begin{lstlisting}[firstnumber=26]
    double surfaceWeight(double otherMass) {
        return otherMass * surfaceGravity();
    }
} // public enum Planet
//Verwendung:
Planet jup = Planet.JUPITER;
double someMass = 10;
double weightOnJupiter = jup.surfaceWeight(someMass);
\end{lstlisting}

Beispiel angelehnt an \cite{orac:enum}
\end{frame}

\begin{frame}{Strukturierte Datentypen}{Structs}
    \begin{itemize}
        \item Fassen mehrere Einzelvariablen zu einer Struktur zusammen
        \item Reine Datenstruktur zum Erfassen von Daten
        \begin{itemize}
            \item Keine Methoden
            \item Keine Zugriffsspezifizierer
        \end{itemize}
        \item Kein separates Konzept in Java $\Rightarrow$ Sonderfall einer Klasse
    \end{itemize}
\end{frame}

\begin{frame}[fragile]{Structs}{Beispiel in C++}
\lstset{style=cpp}
\begin{lstlisting}
struct Date{
    int day;
    int month;
    int year;
    
    Date():day(1)
            ,month(1)
            ,year(1970){}
};
//Verwendung:
struct Date oneDate = {17,3,2019}
oneDate.month=7;
\end{lstlisting}
\end{frame}

\begin{frame}[fragile]{Structs}{Beispiel in Java}
\lstset{style=java}
\begin{lstlisting}
public class DateStruct{
    int day = 1;
    int month = 1;
    int year = 1970;
}
//Verwendung:
DateStruct aDate = new DateStruct();
aDate.day=1;
aDate.month=4;
aDate.year=2019;
\end{lstlisting}
\end{frame}

\begin{frame}{Structs}{Verwendung in Java}
    \begin{alertblock}{Auszug der Java Code Conventions(Siehe \cite{orac:cc})}
    \textbf{10 - Programming Practices}
    
    \textbf{10.1 Providing Access to Instance and Class Variables}
    
    Don't make any instance or class variable public without good reason. Often, instance variables don't need to be explicitly set or gotten-often that happens as a side effect of method calls.
    
    \visible<2>{One example of appropriate public instance variables is the case where the class is essentially a data structure, with no behavior. In other words, if you would have used a \texttt{struct} instead of a class (if Java supported \texttt{struct}), then it's appropriate to make the class's instance variables public.}
    \end{alertblock}
\end{frame}

\begin{frame}{Strukturierte Datentypen}{Unions}
    \begin{itemize}
        \item Enthält mehrere Variablen
        \item Diese teilen sich den gleichen Speicher
        \item Variablen können demnach nicht unabhängig voneinander verändert werden
        \item Streng genommen immer nur ein Member "`aktiv"'
        \begin{itemize}
            \item "`Aktiv"' ist immer das member, auf das zuletzt geschrieben wurde
            \item Lesen von den anderen membern ist undefiniertes Verhalten
            \item Von den meisten Compilern jedoch definiert
        \end{itemize}
        \item Kein (mir bekanntes) äquivalent in Java
    \end{itemize}
\end{frame}

\begin{frame}[fragile]{Unions}{Codebeispiel in C++}
\lstset{style=cpp}
\begin{lstlisting}
Union Example{
    std::int32_t n;     //Belegt 4 Byte
    std::uint16_t s[2]  //Belegt 4 Byte
    std::uint8_t c;     //Belegt 1 Byte
};                      //Gesamte Union belegt 4 Byte
//Verwendung:
Example ex = {0x12345678};  //n wird initialisiert. Lesen von s und c ist undefiniertes Verhalten
ex.s[0] = 0x0011;   //s wird zum aktiven member
std::cout << ex.c << std::endl; //0x11 oder 0x00, Plattformspezifisch
std::cout << ex.n << std::endl; //0x12340011 oder 0x00115678
\end{lstlisting}

Vgl. \cite{msdn:union}, \cite{cppref:union}
\end{frame}