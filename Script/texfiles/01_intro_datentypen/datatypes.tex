\begin{frame}{Datentypen}{Unterscheidung}
	\begin{itemize}
		\item Primitive Datentypen
		\item Objektorientierte Datentypen
		\item Strukturierte Datentypen
	\end{itemize}
\end{frame}

\begin{frame}{Primitive Datentypen}{Beispiele}
	\begin{itemize}
	\item Beispiele:
		\begin{itemize}
			\item int
			\item boolean
			\item char
			\item float
			\item double
			\item long
			\item byte
		\end{itemize}
	\end{itemize}
\end{frame}

\begin{frame}{Primitive Datentypen}{Eigenschaften}
	\begin{itemize}
		\item Speichern "einfache" Daten
		\item Feste Speichergröße
		\item Feste Präzision $\Rightarrow$ Diskret
		\item Fest definierte Ober- und Untergrenze
		\item Menge an Operationen beschränkt
		\begin{itemize}
			\item Und nicht erweiterbar
		\end{itemize}
	\end{itemize}
\end{frame}

\begin{frame}{Primitive Datentypen}{Eigenschaften}
	\begin{tabular}{|c|c|c|c|}
	\hline
	\textbf{Typ} & \textbf{Größe (Bits)} & \textbf{Minimum} & \textbf{Maximum} 
	\\
	\hline
	\visible<+->{\textbf{byte}} & \visible<+->{8} & \visible<+->{$ -128 $} & \visible<+->{$ 127 $}  \\
	\hline
	\textbf{char} & \visible<+->{16} & \visible<+->{$ 0 $} & \visible<+->{$ 2^{16}-1 $} \\
	\hline
	\textbf{short} & \visible<+->{16} & \visible<+->{$ -2^{15} $} & \visible<+->{$ 2^{15}-1 $} \\
	\hline
	\textbf{int} & \visible<+->{32} & \visible<+->{$ -2^{31} $} & \visible<+->{$ 2^{31}-1 $} \\
	\hline
	\textbf{long} & \visible<+->{64} & \visible<+->{$ -2^{63} $} & \visible<+->{$ 2^{63}-1 $} \\
	\hline
	\textbf{float} & \visible<+->{32} & \visible<+->{$ \pm 1.4\text{E-}45 $} & \visible<+->{$ \pm 3.4\text{E+}38 $}\\
	\hline
	\textbf{double} & \visible<+->{64} & \visible<+->{$ \pm 4.9\text{E-}324 $} & \visible<+->{$ \pm 1.7\text{E+}324 $}\\
	\hline
	\textbf{boolean} & \visible<+->{Undefiniert} & \multicolumn{2}{c|}{\visible<+->{Nur \tt{true} und \tt{false}}} \\
	\hline
	\end{tabular}
\end{frame}